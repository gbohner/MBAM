
\documentclass[10pt]{amsart}

\usepackage{amssymb}
%\input{macros}

\begin{document}

\title{Reducibility and robustness in allosteric macromolecules}

\author{Gerg\H{o} Bohner\and Gaurav Venkataraman}

\date{\vspace{-.1in}}

\begin{abstract}

Macromolecules transfer information by sensing stimuli and transducing this stimulus information into conformational changes.  These conformational changes are often allosteric: one region of the macromolecule undergoes structural rearrangement in response to stimulus applied to a different region of the same molecule.  Here we address the issue of sensitivity of allosteric macromolecules to their underlying biochemical and coupling parameters.  We demonstrate that allosteric macromolecules are ?sloppy? systems, whose function depends only on a few combinations of parameters. We investigate the consequences of sloppiness in the context of several quantitative, mechanistic allosteric membrane protein gating models.  The parameters of these mechanistic models are seen to be fundamentally non-identifiable with respect to functional data. The mechanistic models are therefore reducible: phenomenological mathematical models with fewer parameters will reproduce the desired functional behavior equally well.  We propose that the particular manner in which the mechanistic model parameters are non-identifiable constitutes an important prediction about the macromolecule.  In particular, the manner in which the model is non-identifiable determines how the mechanistic model is reduced into its phenomenological counterpart.  We demonstrate that this reduction sheds light onto the functional mechanism of the macromolecule.  We argue that thinking of non-identifiable mechanistic models in terms of the reduced models that they give rise to is a productive way to understand the function of biological macromolecules at a mechanistic level.

\end{abstract}

\maketitle

\section{Introduction}

Macromolecules often perform signal transduction by undergoing conformational changes in response to targeted stimuli.  The mechanisms by which a small targeted stimuli is able to regulate macromolecular behavior at locations structurally distant from the active site of stimulation is often referred to as ?allostery,? and has been the subject of considerable study.  An important tool for investigating these allosteric regulatory mechanisms is the Monod-Wyman-Changeux (MWC) model, which provides  a physical-chemical interpretation of indirect regulation in terms of the geometry of the molecule.  The MWC model has been employed to garner insights from seemingly disparate aspects of biology, including: ligand-receptor binding, chemotaxis, and ligand-gated ion channels.



Operationally, any given MWC model represents a candidate hypothesis for how allosteric conformational change occurs.  If a model is not able to quantitatively fit available data, it is rejected.  For models that agree with the data, the fits provide estimates of the underlying model parameters.  These parameters are biophysically meaningful properties of interest that cannot be measured directly, such as binding affinities, strength of cooperative interactions, and kinetic rate constants.  Therefore, much effort has been put forth towards determining the parameters that lead to ``best fits?? of models to data.  Recent work has focused on accurate confidence intervals for parameter estimation, and demonstrated that even simple mechanistic models exhibit non-identifiability: multiple parameter values fit available data equally well.



The issue of parameter non-identifiability has also been the subject of much recent work in the statistical physics community.  It has been observed that many systems biology models are non-identifiable in such a way that huge swaths of parameter space have no effect on model output. This phenomena is termed ?sloppiness.?  Naively, the presence of sloppiness may seem to doom modeling efforts.  On the contrary, it has been argued that sloppy parameter spaces underlie predictability in scientific theories: macroscopic scientific theories have been observed to be derivable from microscopic theories by removing the irrelevant sloppy directions.  Subsequent work along these lines has resulted in the Manifold Boundary Approximation Method (MBAM), which allows for the algorithmic generation of coarse-grained, identifiable models from sloppy, non-identifiable ones.  Crucially, MBAM does not result in ?back box? models, but instead generates coarse-grained models which are explicitly expressed in terms of the microscopic parameters.



Here, we demonstrate that a number of allosteric macromolecules are sloppy.  We show that we can algorithmically generate identifiable coarse-grained models from non-identifiable sloppy ones.  Given these results, we ultimately ask: do the identifiable coarse-grained models shed light onto underlying allosteric mechanism?

\end{document}
