\documentclass{article}

\usepackage{verbatim}
\usepackage{amssymb}
\usepackage{amsmath}
%\input{macros}

%\setlength{\textwidth}{\paperwidth}
%\addtolength{\textwidth}{-2in}
%\calclayout

\newcommand\Ld{\sqrt[1/4]{L_0}D}

\newcommand\Lc{\sqrt[1/4]{L_0}C}


\newcommand\Le{\frac{E}{\sqrt[1/4]{L_0}}}

\newcommand\Po{P_o}

\usepackage{float}
\newfloat{suppfig}{tbh}{losf}
\floatname{suppfig}{Supplementary Figure}

\newcommand\po{P_o}

\newcommand\lpo{\log(\Po)}

\newcommand\ca{\rm{Ca}^{2+}}

\newcommand\kk{\rm{K}^+}

\newcommand\kd{K_D}

\newcommand\pio{p(\mathcal{I},\mathcal{O})}

\newcommand{\ltwo}{\log_2}

\usepackage{tikz}
%\usepackage{tikz-cd}
\usepackage{pgfplots}
\usepackage[utf8]{inputenc}
\usepackage[upright]{fourier}
\usetikzlibrary{arrows,automata,positioning,matrix,cd,babel,matrix,arrows,decorations.pathmorphing,intersections}
\usetikzlibrary{intersections}

\usepackage[many]{tcolorbox}

\usepackage{graphicx,subcaption}




\pgfplotsset{compat=1.12}

\usepackage{tikz}
\usetikzlibrary{arrows,positioning,calc}
\usepackage{tikz}
\usepackage{cancel}
\usepackage{color}
\usetikzlibrary{shapes.geometric, arrows}
%\usepackage{subfigure}

\newcommand{\zj}{z_J}
\newcommand{\zl}{z_L}
\newcommand{\Lo}{L_0}
\newcommand{\Jo}{J_0}
%\newcommand{\kd}{K_d}

\usetikzlibrary{plotmarks}




%\newcommand{\ca}{[\rm Ca^{2+}]}


%% http://tex.stackexchange.com/questions/55068/is-there-a-tikz-equivalent-to-the-pstricks-ncbar-command
\tikzset{
    ncbar angle/.initial=90,
    ncbar/.style={
        to path=(\tikztostart)
        -- ($(\tikztostart)!#1!\pgfkeysvalueof{/tikz/ncbar angle}:(\tikztotarget)$)
        -- ($(\tikztotarget)!($(\tikztostart)!#1!\pgfkeysvalueof{/tikz/ncbar angle}:(\tikztotarget)$)!\pgfkeysvalueof{/tikz/ncbar angle}:(\tikztostart)$)
        -- (\tikztotarget)
    },
    ncbar/.default=0.5cm,
}

\tikzset{square left brace/.style={ncbar=0.5cm}}
\tikzset{square right brace/.style={ncbar=-0.5cm}}

\tikzset{round left paren/.style={ncbar=0.5cm,out=120,in=-120}}
\tikzset{round right paren/.style={ncbar=0.5cm,out=60,in=-60}}

\newtcolorbox{cross}{blank,breakable,parbox=false,
  overlay={\draw[red,line width=5pt] (interior.south west)--(interior.north east);
    \draw[red,line width=5pt] (interior.north west)--(interior.south east);}}


\newcommand\hcancel[2][black]{\setbox0=\hbox{#2}
\rlap{\raisebox{.45\ht0}{\textcolor{#1}{\rule{\wd0}{2pt}}}}#2}

%\newcommand{\kd}{K_d}

\usetikzlibrary{arrows,positioning} 

\newcommand\rcancel[2][red]{\renewcommand\CancelColor{\color{#1}}\xcancel{#2}}

\newcommand{\no}{N_{open}}
\newcommand{\ono}{\overline{\no}}
\newcommand{\var}{\sigma^2_{\rm N_{open}}}
\newcommand{\lo}{L_0}
\newcommand{\jo}{J_0}
%\newcommand{\zl}{z_L}
%\newcommand{\zj}{z_J}
%\\renewcommand{\b}[1]{\left( #1 \right)

 \definecolor{mediumcandyapplered}{rgb}{0.89, 0.02, 0.17}
 
 \pgfplotsset{%
    axis line origin/.style args={#1,#2}{
        x filter/.append code={ % Check for empty or filtered out numbers
            \ifx\pgfmathresult\empty\else\pgfmathparse{\pgfmathresult-#1}\fi
        },
        y filter/.append code={
            \ifx\pgfmathresult\empty\else\pgfmathparse{\pgfmathresult-#2}\fi
        },
        xticklabel=\pgfmathparse{\tick+#1}\pgfmathprintnumber{\pgfmathresult},
        yticklabel=\pgfmathparse{\tick+#2}\pgfmathprintnumber{\pgfmathresult}
    }
}


% Definition for Calcium concentrations and appropriate labels to be used in foreach loops over generated CSV files
 \edef\CaSTR{{0.0um}/0,{0.7um}/0.7,{4.0um}/4,{12.0um}/12,{22.0um}/22,{55.0um}/55,{70.0um}/70,{95.0um}/95}

 % Enables pgfplots to reset the color cycle within a plot via \pgfplotsset{cycle list set=0}
% Credit to Jake in thread http://tex.stackexchange.com/questions/74315/command-to-set-cycle-list-position

\makeatletter
\def\pgfplots@getautoplotspec into#1{%
    \begingroup
    \let#1=\pgfutil@empty
    \pgfkeysgetvalue{/pgfplots/cycle multi list/@dim}\pgfplots@cycle@dim
    %
    \let\pgfplots@listindex=\pgfplots@numplots
    %%% Start new code
    \pgfkeysgetvalue{/pgfplots/cycle list set}\pgfplots@listindex@set
    \ifx\pgfplots@listindex@set\pgfutil@empty
    \else 
        \c@pgf@counta=\pgfplots@listindex
        \c@pgf@countb=\pgfplots@listindex@set
        \advance\c@pgf@countb by -\c@pgf@counta
        \globaldefs=1\relax
        \edef\setshift{%
            \noexpand\pgfkeys{
                /pgfplots/cycle list shift=\the\c@pgf@countb,
                /pgfplots/cycle list set=
            }
        }%
        \setshift%
    \fi
    %%% End new code    
    \pgfkeysgetvalue{/pgfplots/cycle list shift}\pgfplots@listindex@shift
    \ifx\pgfplots@listindex@shift\pgfutil@empty
    \else
        \c@pgf@counta=\pgfplots@listindex\relax
        \advance\c@pgf@counta by\pgfplots@listindex@shift\relax
        \ifnum\c@pgf@counta<0
            \c@pgf@counta=-\c@pgf@counta
        \fi
        \edef\pgfplots@listindex{\the\c@pgf@counta}%
    \fi
    \ifnum\pgfplots@cycle@dim>0
        % use the 'cycle multi list' feature.
        %
        % it employs a scalar -> multiindex map like
        % void fromScalar( size_t d, size_t scalar, size_t* Iout, const size_t* N )
        % {
        %   size_t ret=scalar;
        %   for( int i = d-1; i>=0; --i ) {
        %       Iout[i] = ret % N[i];
        %       ret /= N[i];
        %   }
        % }
        % to get the different indices into the cycle lists.
        %-------------------------------------------------- 
        \c@pgf@counta=\pgfplots@cycle@dim\relax
        \c@pgf@countb=\pgfplots@listindex\relax
        \advance\c@pgf@counta by-1
        \pgfplotsloop{%
            \ifnum\c@pgf@counta<0
                \pgfplotsloopcontinuefalse
            \else
                \pgfplotsloopcontinuetrue
            \fi
        }{%
            \pgfkeysgetvalue{/pgfplots/cycle multi list/@N\the\c@pgf@counta}\pgfplots@cycle@N
            % compute list index:
            \pgfplotsmathmodint{\c@pgf@countb}{\pgfplots@cycle@N}%
            \divide\c@pgf@countb by \pgfplots@cycle@N\relax
            %
            \expandafter\pgfplots@getautoplotspec@
                \csname pgfp@cyclist@/pgfplots/cycle multi list/@list\the\c@pgf@counta @\endcsname
                {\pgfplots@cycle@N}%
                {\pgfmathresult}%
            \t@pgfplots@toka=\expandafter{#1,}%
            \t@pgfplots@tokb=\expandafter{\pgfplotsretval}%
            \edef#1{\the\t@pgfplots@toka\the\t@pgfplots@tokb}%
            \advance\c@pgf@counta by-1
        }%
    \else
        % normal cycle list:
        \pgfplotslistsize\autoplotspeclist\to\c@pgf@countd
        \pgfplots@getautoplotspec@{\autoplotspeclist}{\c@pgf@countd}{\pgfplots@listindex}%
        \let#1=\pgfplotsretval
    \fi
    \pgfmath@smuggleone#1%
    \endgroup
}

\pgfplotsset{
    cycle list set/.initial=
}
\makeatother

\pgfplotsset{
    every axis plot post/.style={
        line join=round
    },
    every axis/.append style={font=\large},
    every tick label/.append style={font=\large}
}

\usepackage{pgfplotstable}
\usepackage{booktabs}

\begin{document}


%%%%%%%%%%%%%%%%%%%%%%%%%%%%%%%%%%%%%%%%%%%%%%%%%%%%%%%%%%%%%%%%%%%%%%%%%%%%%%%%%%%%%%%%%%%%%%%%%%%%%%%%%%%%%%%%%%%%%%%%%%%%%%%%%%%%%%%%%%%%%%%
%%%%%%%%%%%%%%%%%%%%%%%%%%%%%%%%%%%%%%%%%%%%% FIGURE 1 %%%%%%%%%%%%%%%%%%%%%%%%%%%%%%%%%%%%%%%%%%%%%%%%%%%%%%%%%%%%%%%%%%%%%%%%%%%%%%%%%%%%%%%%
%%%%%%%%%%%%%%%%%%%%%%%%%%%%%%%%%%%%%%%%%%%%%%%%%%%%%%%%%%%%%%%%%%%%%%%%%%%%%%%%%%%%%%%%%%%%%%%%%%%%%%%%%%%%%%%%%%%%%%%%%%%%%%%%%%%%%%%%%%%%%%%


\begin{figure}


    %%%%%%%%%%%%%%%%%%%%%%%%%%%%%%%%%%%%%%%%%%%%%%%%%%%%%%%%%%%%%%%%%%%%%%%%%%%%%%%%%%%%%%%%%%%%%%%%%%%%%%%%%%%%%%%%%%%%%%%%%%%%%%%%%%%%%%%%%%%
    %%%%%%%%%%%%%%%%%%%%%%%%%%%%%%%%%%%%%%%%%%%%% FIGURE 1 A %%%%%%%%%%%%%%%%%%%%%%%%%%%%%%%%%%%%%%%%%%%%%%%%%%%%%%%%%%%%%%%%%%%%%%%%%%%%%%%%%%
    %%%%%%%%%%%%%%%%%%%%%%%%%%%%%%%%%%%%%%%%%%%%%%%%%%%%%%%%%%%%%%%%%%%%%%%%%%%%%%%%%%%%%%%%%%%%%%%%%%%%%%%%%%%%%%%%%%%%%%%%%%%%%%%%%%%%%%%%%%%

	\begin{center}
	\begin{subfigure}[b]{0.6\linewidth}
	\centering
	\resizebox{\linewidth}{!}{

\begin{tikzpicture}
    % Figure letter:
    \node[draw=none, fill=none] at (0,10) {\Huge A};


    \node[draw=none,fill=none] (R) at (.75,3.5) {\Huge R};
    \node[draw=none,fill=none] (A) at (.75,0.5) {\Huge A};
    \draw [<- >, very thick] (A) to node[anchor=west] {\huge{$J_0^{(z_J)}$}} (R);
    %\draw[<->] (A) to {v} (RA);% {V};
    %\draw[->, bend right=22.5] (\from) to node[fill=white] {$T_{\from \to}$}
    \draw [black, very thick] (0,0) to [square left brace ] (0,4);
    \draw [black, very thick] (1.5,0) to [square right brace] (1.5,4);

    
    \draw [black, very thick] (10,0) to [square left brace ] (10,4);
    \draw [black, very thick] (11.5,0) to [square right brace] (11.5,4);
        \node[draw=none,fill=none] (X) at (10.75,3.5) {\Huge X};
    \node[draw=none,fill=none] (Xca) at (10.75,0.5) {{\huge{X}$\cdot\rm \large{Ca^{2+}}$}};
    \draw [<->, very thick, name=K] (X) to node[anchor=west] {\huge{$K_d$}} (Xca);
    
    \draw [black, very thick] (5,5) to [square left brace ] (5,9);
    \draw [black, very thick] (6.5,5) to [square right brace] (6.5,9);
    
    \node[draw=none,fill=none] (C) at (5.75,8.5) {\Huge C};
    \node[draw=none,fill=none] (O) at (5.75,5.5) {\huge{O}};
    \draw [<->, very thick, name=L] (C) to node[anchor=west] {\huge ${L}_0^{{{(z_L)}}}$} (O);
        \node [draw=none,fill=none] (four) at (2.3,0) {\huge 4};
    
	    \node [draw=none,fill=none] (four) at (12.3,0) {\huge 4};
    
    %D should be at (2.5+.75,7) $5.75+2.5=8.25 10.75-2.5 = 
    
    %guide to locations:
    	%J_v node is at (1.8,2)
        %K node is at (10.8,2)
        %L node is at (6.25,6.8)
        %D node is at (2.5,6)
        %C node is at (9,6)
        %E node is at 5.75,2
    
    
    \node[draw=none,fill=none,rotate=45] (D) at (2.5,6) {\Huge $ {D} $};
    
    \node[draw=none,fill=none,rotate=45, left of=D, xshift=.2cm] (<D) at (2.5,6) {\Huge $<$};
    
     \node[draw=none,fill=none,rotate=45, right of=D, xshift=-.2cm] (D>) at (2.5,6) {\Huge $>$};    
     
         \node[draw=none,fill=none,rotate=-45] (C) at (9,6) {\Huge $   {C} $};

         
    \node[draw=none,fill=none,rotate=45, above of=C, yshift=-.2cm] (<C) at (9,6) {\Huge $\wedge$};
    
     \node[draw=none,fill=none,rotate=45, below of=C, yshift=.2cm] (C>) at (9,6) {\Huge $\vee$};    

    
    \node[draw=none,fill=none] (E) at (5.75,2) {\Huge $ {E} $};
    
    
        \node[draw=none,fill=none, left of=E, xshift=.2cm] (<E) at (5.75,2) {\Huge $<$};
    
     \node[draw=none,fill=none, right of=E, xshift=-.2cm] (E>) at (5.75,2) {\Huge $>$};    
    
 %   \draw [-, line width=0.7mm, red, bend left=15,dashed] (6.25,6.8) to node[anchor=east,font=\bf] {\huge$\phi_1=\sqrt[4]{L_0}E$} (5.75,2);
    
%    \draw [-, line width=0.7mm, red, dashed] (6.25,7.25) .. controls +(up:6cm) and +(left:1cm) .. node[above,sloped,font=\bf] {\huge$\mathbf \phi_2=\sqrt[4]{L_0}D$} (2.5,6);
    
 %       \draw [-, line width=0.7mm, red, dashed] (6.25,7.25) .. controls +(up:6cm) and +(right:1cm) .. node[above,sloped,font=\bf] {\huge$\phi_3=\sqrt[4]{L_0}C$} (9,6);
\end{tikzpicture}
}

\end{subfigure}
\end{center}


\bigskip


    %%%%%%%%%%%%%%%%%%%%%%%%%%%%%%%%%%%%%%%%%%%%%%%%%%%%%%%%%%%%%%%%%%%%%%%%%%%%%%%%%%%%%%%%%%%%%%%%%%%%%%%%%%%%%%%%%%%%%%%%%%%%%%%%%%%%%%%%%%%
    %%%%%%%%%%%%%%%%%%%%%%%%%%%%%%%%%%%%%%%%%%%%% FIGURE 1 B %%%%%%%%%%%%%%%%%%%%%%%%%%%%%%%%%%%%%%%%%%%%%%%%%%%%%%%%%%%%%%%%%%%%%%%%%%%%%%%%%%
    %%%%%%%%%%%%%%%%%%%%%%%%%%%%%%%%%%%%%%%%%%%%%%%%%%%%%%%%%%%%%%%%%%%%%%%%%%%%%%%%%%%%%%%%%%%%%%%%%%%%%%%%%%%%%%%%%%%%%%%%%%%%%%%%%%%%%%%%%%%


\hspace{-1.5cm}
\begin{subfigure}[b]{0.6\linewidth}
	\resizebox{\linewidth}{!}{
\begin{tikzpicture}
\node[draw=none, fill=none] at (0,6) {\Large B};
\begin{axis}
       [
         axis line style = { draw = none },
         xtick           = {-100,0,160}         ,
         ytick           = {0,.5,1}         ,
         tick pos        = left           ,
         xlabel={V (mv)},
         ylabel = {$P_o$},
       ]


\foreach \i/\j in \CaSTR
{
\addplot +[only marks] table [x index=0, y index=1, col sep=comma] {../CSV/figure_1b_panel1_ca\i.csv};\label{\j}
}
\pgfplotsset{cycle list set=0}
\foreach \i/\j in \CaSTR
{
\addplot +[no markers] table [x index=0, y index=1, col sep=comma, mark=circle]{../CSV/figure_1b_panel2_ca\i.csv};
}
%\addplot table [x index=0, y index=0, col sep=comma] {figure_3b_panel1_paramzJ.csv};
%\addplot table [x=trila, y=st]
    %     \node [left] at (axis cs:  6,6) {$32hhhh$};
\end{axis}

\end{tikzpicture}
}

\end{subfigure}
    %%%%%%%%%%%%%%%%%%%%%%%%%%%%%%%%%%%%%%%%%%%%%%%%%%%%%%%%%%%%%%%%%%%%%%%%%%%%%%%%%%%%%%%%%%%%%%%%%%%%%%%%%%%%%%%%%%%%%%%%%%%%%%%%%%%%%%%%%%%
    %%%%%%%%%%%%%%%%%%%%%%%%%%%%%%%%%%%%%%%%%%%%% FIGURE 1 C %%%%%%%%%%%%%%%%%%%%%%%%%%%%%%%%%%%%%%%%%%%%%%%%%%%%%%%%%%%%%%%%%%%%%%%%%%%%%%%%%%
    %%%%%%%%%%%%%%%%%%%%%%%%%%%%%%%%%%%%%%%%%%%%%%%%%%%%%%%%%%%%%%%%%%%%%%%%%%%%%%%%%%%%%%%%%%%%%%%%%%%%%%%%%%%%%%%%%%%%%%%%%%%%%%%%%%%%%%%%%%%
\begin{subfigure}[b]{0.6\linewidth}
	\resizebox{\linewidth}{!}{

\begin{tikzpicture}
\node[draw=none, fill=none] at (0,6) {\Large C};
\begin{axis}
       [
         axis line style = { draw = none },
         xtick           = {-100,0,160}         ,
         ytick           = {0,-3,-6}         ,
         tick pos        = left           ,
         xlabel={V (mv)},
         ylabel = {$\log(P_o)$},
       ]

\pgfplotsset{cycle list set=0}
\foreach \i/\j in \CaSTR
{
\addplot +[only marks] table [x index=0, y index=1, col sep=comma] {../CSV/figure_1c_panel1_ca\i.csv};
}
\pgfplotsset{cycle list set=0}
\foreach \i/\j in \CaSTR
{
\addplot +[no markers] table [x index=0, y index=1, col sep=comma, mark=circle]{../CSV/figure_1c_panel2_ca\i.csv};
}

%\addplot table [x index=0, y index=0, col sep=comma] {figure_3b_panel1_paramzJ.csv};
%\addplot table [x=trila, y=st]
    %     \node [left] at (axis cs:  6,6) {$32hhhh$};
\end{axis}

\end{tikzpicture}
}

\end{subfigure}

\captionsetup{width=1.1\linewidth}
\caption{Synthetic steady-state data.  (A) Schematic of the general allosteric gating mechanism used to generate synthetic data.  The steady state properties of this model are fully described by eight parameters, three of which define the allosteric interactions (C, D, E) and the remaining five define the equilibrium constants (J,K,L) via $J=f(J_0,\zj), \ K=f(\kd), \ L=f(\Lo, \ \zl).$  (B, C) $P_o-V$ and $\log(P_o)-V$ relationships generated from the scheme in (A) for different $\ca$(in $\mu m:$ 0 (\ref{0}); 0.7 (\ref{0.7}); 4 (\ref{4}); 12 (\ref{12}); 22 (\ref{22}); 55 (\ref{55}); 70 (\ref{70}); 95 (\ref{95})) using previously published best-fit parameters .  Each curve contains 26 data points, sampled from the underlying binding curve (solid lines).}




\end{figure}

\pagebreak
%%%%%%%%%%%%%%%%%%%%%%%%%%%%%%%%%%%%%%%%%%%%%%%%%%%%%%%%%%%%%%%%%%%%%%%%%%%%%%%%%%%%%%%%%%%%%%%%%%%%%%%%%%%%%%%%%%%%%%%%%%%%%%%%%%%%%%%%%%%%%%%
%%%%%%%%%%%%%%%%%%%%%%%%%%%%%%%%%%%%%%%%%%%%%%%%%%%%%%%%%%%%%%%%%%%%%%%%%%%%%%%%%%%%%%%%%%%%%%%%%%%%%%%%%%%%%%%%%%%%%%%%%%%%%%%%%%%%%%%%%%%%%%%
%%%%%%%%%%%%%%%%%%%%%%%%%%%%%%%%%%%%%%%%%%%%% FIGURE 2 %%%%%%%%%%%%%%%%%%%%%%%%%%%%%%%%%%%%%%%%%%%%%%%%%%%%%%%%%%%%%%%%%%%%%%%%%%%%%%%%%%%%%%%%
%%%%%%%%%%%%%%%%%%%%%%%%%%%%%%%%%%%%%%%%%%%%%%%%%%%%%%%%%%%%%%%%%%%%%%%%%%%%%%%%%%%%%%%%%%%%%%%%%%%%%%%%%%%%%%%%%%%%%%%%%%%%%%%%%%%%%%%%%%%%%%%
%%%%%%%%%%%%%%%%%%%%%%%%%%%%%%%%%%%%%%%%%%%%%%%%%%%%%%%%%%%%%%%%%%%%%%%%%%%%%%%%%%%%%%%%%%%%%%%%%%%%%%%%%%%%%%%%%%%%%%%%%%%%%%%%%%%%%%%%%%%%%%%

\begin{figure}


    %%%%%%%%%%%%%%%%%%%%%%%%%%%%%%%%%%%%%%%%%%%%%%%%%%%%%%%%%%%%%%%%%%%%%%%%%%%%%%%%%%%%%%%%%%%%%%%%%%%%%%%%%%%%%%%%%%%%%%%%%%%%%%%%%%%%%%%%%%%
    %%%%%%%%%%%%%%%%%%%%%%%%%%%%%%%%%%%%%%%%%%%%% FIGURE 2 A %%%%%%%%%%%%%%%%%%%%%%%%%%%%%%%%%%%%%%%%%%%%%%%%%%%%%%%%%%%%%%%%%%%%%%%%%%%%%%%%%%
    %%%%%%%%%%%%%%%%%%%%%%%%%%%%%%%%%%%%%%%%%%%%%%%%%%%%%%%%%%%%%%%%%%%%%%%%%%%%%%%%%%%%%%%%%%%%%%%%%%%%%%%%%%%%%%%%%%%%%%%%%%%%%%%%%%%%%%%%%%%
 \hspace{-.6cm}
\begin{subfigure}[b]{0.6\linewidth}
	\resizebox{\linewidth}{!}{
\begin{tikzpicture}
\node[draw=none, fill=none] at (0,6) {\Large A};
\begin{axis}
       [
         axis line style = { draw = none },
         xtick           = {-100,0,160}         ,
         ytick           = {0,.5,1}         ,
         tick pos        = left           ,
         xlabel={V (mv)},
         ylabel = {$P_o$},
       ]


\pgfplotsset{cycle list set=0}
\foreach \i/\j in \CaSTR
{
\addplot +[no markers] table [x index=0, y index=1, col sep=comma, mark=circle]{../CSV/figure_2a_panel2_ca\i.csv};
}
\pgfplotsset{cycle list set=0}
\foreach \i/\j in \CaSTR
{
\addplot +[only marks] plot [error bars/.cd, y dir = both, y explicit] table [x index=0, y index=1, y error index=2, col sep=comma] {../CSV/figure_2a_panel1_ca\i.csv};\label{\j}
}


%\addplot table [x index=0, y index=0, col sep=comma] {figure_3b_panel1_paramzJ.csv};
%\addplot table [x=trila, y=st]
    %     \node [left] at (axis cs:  6,6) {$32hhhh$};
\end{axis}

\end{tikzpicture}
}

\end{subfigure}
    %%%%%%%%%%%%%%%%%%%%%%%%%%%%%%%%%%%%%%%%%%%%%%%%%%%%%%%%%%%%%%%%%%%%%%%%%%%%%%%%%%%%%%%%%%%%%%%%%%%%%%%%%%%%%%%%%%%%%%%%%%%%%%%%%%%%%%%%%%%
    %%%%%%%%%%%%%%%%%%%%%%%%%%%%%%%%%%%%%%%%%%%%% FIGURE 2 B %%%%%%%%%%%%%%%%%%%%%%%%%%%%%%%%%%%%%%%%%%%%%%%%%%%%%%%%%%%%%%%%%%%%%%%%%%%%%%%%%%
    %%%%%%%%%%%%%%%%%%%%%%%%%%%%%%%%%%%%%%%%%%%%%%%%%%%%%%%%%%%%%%%%%%%%%%%%%%%%%%%%%%%%%%%%%%%%%%%%%%%%%%%%%%%%%%%%%%%%%%%%%%%%%%%%%%%%%%%%%%%
\begin{subfigure}[b]{0.6\linewidth}
\begin{tikzpicture}
\node[draw=none, fill=none] at (-2,3.75) {\large B};
    \matrix(dict)[matrix of nodes,%below=of game,
        nodes={align=center,text width=1.5cm},
        row 1/.style={anchor=south},
        column 1/.style={nodes={text width=.9cm,align=right}}
    ]{
        $\theta_i$ & Base & Fit\\
        $L_0$ & $2.2\times 10^{-6}$ & $7.7\times 10^{-37}$\\
        $z_L$ & 0.42 & $8.8 \times 10^{-33}$\\
        $J_0$ & 0.10 & $0.066$\\
        $z_J$ & 0.58 & $0.59$\\
        $K_d$ & $3.9\times10^{-5}$ & $2.7\times10^{-5}$\\
        $C$ & 6.16 & $1.7\times 10^{8}$ \\
        $D$ & 30.4 & $2.4\times10^{9}$\\
        $E$ & 2.0 & $4.5\times 10^{-8}$\\
    };
    \draw(dict-1-1.south west)--(dict-1-3.south east);
    \draw(dict-1-2.north west)--(dict-9-2.south west);
    \draw(dict-1-2.north east)--(dict-9-2.south east);


\begin{comment}
    \pgfplotstabletypeset[
    header=true,
    col sep=comma,
    display columns/0/.style={column name={}},
    display columns/1/.style={
        column name={Base}
    },
    display columns/2/.style={
        column name={Diverging}
    },
    create on use/newcol/.style={
        create col/set list={$L_0$, $z_L$, $J_0$, $z_J$, $K_d$, $C$, $D$, $E$}
    },
    columns/newcol/.style={string type},
    columns={newcol,Base,Diverging},
    every head row/.style={
        before row=\toprule,
        after row=\midrule},
    every last row/.style={
        after row=\bottomrule}
    ]{../CSV/figure_2b_panel1.csv}

    \end{comment}
\end{tikzpicture}
\end{subfigure}
% Data is in {../CSV/figure_2b_panel1.csv}


\vspace{1cm}

 %%%%%%%%%%%%%%%%%%%%%%%%%%%%%%%%%%%%%%%%%%%%%%%%%%%%%%%%%%%%%%%%%%%%%%%%%%%%%%%%%%%%%%%%%%%%%%%%%%%%%%%%%%%%%%%%%%%%%%%%%%%%%%%%%%%%%%%%%%%
    %%%%%%%%%%%%%%%%%%%%%%%%%%%%%%%%%%%%%%%%%%%%% FIGURE 2 C/1 %%%%%%%%%%%%%%%%%%%%%%%%%%%%%%%%%%%%%%%%%%%%%%%%%%%%%%%%%%%%%%%%%%%%%%%%%%%%%%%%%%
    %%%%%%%%%%%%%%%%%%%%%%%%%%%%%%%%%%%%%%%%%%%%%%%%%%%%%%%%%%%%%%%%%%%%%%%%%%%%%%%%%%%%%%%%%%%%%%%%%%%%%%%%%%%%%%%%%%%%%%%%%%%%%%%%%%%%%%%%%%%

\hspace{-.6cm}
    \begin{subfigure}[b]{0.6\linewidth}
    \resizebox{\linewidth}{!}{

\begin{tikzpicture}
\node[draw=none, fill=none] at (0,6.6) {\Large C};
\begin{axis}
       [
         axis line style = { draw = none },
         %ymode=log,
         xtick          = \empty,
         %ytick           = {-10, 0, 10}         ,
         ymin = -50,
         ymax = 0 ,
         tick pos        = left           ,
         xlabel={},
         %xticklabels    = {{Base}, {Diverging}},
         ylabel = {$\log_{10}(L_0)$},
       ]
       


\def\cursf{a}

\addplot +[only marks, color=black, mark=star] table [x index=0, y index=1, col sep=comma] {../CSV/figure_2\cursf_panel3.csv};
\addplot +[only marks, color=red, mark=diamond, mark size=8] table [x index=0, y index=1, col sep=comma] {../CSV/figure_2\cursf_panel1.csv};
%\addplot +[only marks, color=black, mark=diamond*, mark size=14] table [x index=0, y index=1, col sep=comma] {../CSV/figure_2\cursf_panel2.csv};

\end{axis}

\end{tikzpicture}
}

\end{subfigure}
%%%%%%%%%%%%%%%%%%%%%%%%%%%%%%%%%%%%%%%%%%%%%%%%%%%%%%%%%%%%%%%%%%%%%%%%%%%%%%%%%%%%%%%%%%%%%%%%%%%%%%%%%%%%%%%%%%%%%%%%%%%%%%%%%%%%%%%%%%%
    %%%%%%%%%%%%%%%%%%%%%%%%%%%%%%%%%%%%%%%%%%%%% FIGURE 2 C/2 %%%%%%%%%%%%%%%%%%%%%%%%%%%%%%%%%%%%%%%%%%%%%%%%%%%%%%%%%%%%%%%%%%%%%%%%%%%%%%%%%%
    %%%%%%%%%%%%%%%%%%%%%%%%%%%%%%%%%%%%%%%%%%%%%%%%%%%%%%%%%%%%%%%%%%%%%%%%%%%%%%%%%%%%%%%%%%%%%%%%%%%%%%%%%%%%%%%%%%%%%%%%%%%%%%%%%%%%%%%%%%%
    \begin{subfigure}[b]{0.6\linewidth}
    \centering
    \resizebox{\linewidth}{!}{

\begin{tikzpicture}
\node[draw=none, fill=none] at (0,6.6) {\Large D};
\begin{axis}
       [
         axis line style = { draw = none },
         %ymode=log,
         xtick          = \empty,
         %ytick           = {-10, 0, 10}         ,
         ymin = -5,
         ymax = 15,
         tick pos        = left           ,
         xlabel={},
         %xticklabels     = {{Base}, {Diverging}},
         ylabel = {$\log_{10}(D)$},
       ]
       

\def\cursf{c}

\addplot +[only marks, color=black, mark=star] table [x index=0, y index=1, col sep=comma] {../CSV/figure_2\cursf_panel3.csv};
\addplot +[only marks, color=red, mark=diamond, mark size=8] table [x index=0, y index=1, col sep=comma] {../CSV/figure_2\cursf_panel1.csv};
%\addplot +[only marks, color=black, mark=diamond*, mark size=14] table [x index=0, y index=1, col sep=comma] {../CSV/figure_2\cursf_panel2.csv};

\end{axis}

\end{tikzpicture}
}

\end{subfigure}

\hspace{.6cm}
\captionsetup{width=1.15\linewidth}
\caption{An illustration of non-identifiability.  (A). Data generated with `base' parameters, well-fit with `Fit' parameters.  Error bars are $10\%$ from data value.  (B). Parameters which generated the data (`base') and the solid lines ('fit') shown in (A).  (C,D).  Log of fitted $\lo, D$ (respectively) values to 100 noisy synthetic $\Po$ datasets generated from `Base' parameter values.  Values span many orders of magnitude. x-axis spread is for ease of visualization, red diamond surrounds the true `base' parameter.}
\end{figure}

\pagebreak



\bigskip

\bigskip

\bigskip



%%%%%%%%%%%%%%%%%%%%%%%%%%%%%%%%%%%%%%%%%%%%%%%%%%%%%%%%%%%%%%%%%%%%%%%%%%%%%%%%%%%%%%%%%%%%%%%%%%%%%%%%%%%%%%%%%%%%%%%%%%%%%%%%%%%%%%%%%%%%%%%
%%%%%%%%%%%%%%%%%%%%%%%%%%%%%%%%%%%%%%%%%%%%%%%%%%%%%%%%%%%%%%%%%%%%%%%%%%%%%%%%%%%%%%%%%%%%%%%%%%%%%%%%%%%%%%%%%%%%%%%%%%%%%%%%%%%%%%%%%%%%%%%
%%%%%%%%%%%%%%%%%%%%%%%%%%%%%%%%%%%%%%%%%%%%% FIGURE 3 %%%%%%%%%%%%%%%%%%%%%%%%%%%%%%%%%%%%%%%%%%%%%%%%%%%%%%%%%%%%%%%%%%%%%%%%%%%%%%%%%%%%%%%%
%%%%%%%%%%%%%%%%%%%%%%%%%%%%%%%%%%%%%%%%%%%%%%%%%%%%%%%%%%%%%%%%%%%%%%%%%%%%%%%%%%%%%%%%%%%%%%%%%%%%%%%%%%%%%%%%%%%%%%%%%%%%%%%%%%%%%%%%%%%%%%%
%%%%%%%%%%%%%%%%%%%%%%%%%%%%%%%%%%%%%%%%%%%%%%%%%%%%%%%%%%%%%%%%%%%%%%%%%%%%%%%%%%%%%%%%%%%%%%%%%%%%%%%%%%%%%%%%%%%%%%%%%%%%%%%%%%%%%%%%%%%%%%%

\begin{figure}


    %%%%%%%%%%%%%%%%%%%%%%%%%%%%%%%%%%%%%%%%%%%%%%%%%%%%%%%%%%%%%%%%%%%%%%%%%%%%%%%%%%%%%%%%%%%%%%%%%%%%%%%%%%%%%%%%%%%%%%%%%%%%%%%%%%%%%%%%%%%
    %%%%%%%%%%%%%%%%%%%%%%%%%%%%%%%%%%%%%%%%%%%%% FIGURE 3 A %%%%%%%%%%%%%%%%%%%%%%%%%%%%%%%%%%%%%%%%%%%%%%%%%%%%%%%%%%%%%%%%%%%%%%%%%%%%%%%%%%
    %%%%%%%%%%%%%%%%%%%%%%%%%%%%%%%%%%%%%%%%%%%%%%%%%%%%%%%%%%%%%%%%%%%%%%%%%%%%%%%%%%%%%%%%%%%%%%%%%%%%%%%%%%%%%%%%%%%%%%%%%%%%%%%%%%%%%%%%%%%
	\hspace{-2cm}
\begin{subfigure}[b]{0.48\linewidth}
\centering
\begin{tikzpicture}
    \node[draw=none, fill=none] at (-7.5, 1.2) {\Large A};
    \matrix(dict)[matrix of nodes,%below=of game,
        nodes={align=center,text width=1.25cm},
        row 1/.style={anchor=south},
        row 2/.style={color=red},
        column 1/.style={nodes={text width=4.0cm,align=right}}
    ]{
        Parameter ($\theta_i)$ & $L_0$ & $z_L$ & $J_0$ & $z_J$ & $K_d$ & $C$ & $D$ & $E$\\
         $P_o$ Relative error (\%)  & $5\times10^{11}$ & $4.0\times10^4$ & $7.4\times10^3$ & $71.80$ & $304.31$ &  $2.8\times10^3$ & $1.8\times10^6$ & $1.7\times10^4$\\
        $\log(P_o)$ Relative error (\%)  & $7.9\times10^3$ & $612.18$ & $262.10$ & $9.50$ & $13.14$ &  $32.88$ & $1.2\times10^3$ & $22.19$\\
    };
    \draw(dict-1-1.south west)--(dict-1-9.south east);
    \draw(dict-1-1.north east)--(dict-3-1.south east);
    % Data is in {../CSV/figure_3a_panel1.csv}
\end{tikzpicture}
\end{subfigure}
    %%%%%%%%%%%%%%%%%%%%%%%%%%%%%%%%%%%%%%%%%%%%%%%%%%%%%%%%%%%%%%%%%%%%%%%%%%%%%%%%%%%%%%%%%%%%%%%%%%%%%%%%%%%%%%%%%%%%%%%%%%%%%%%%%%%%%%%%%%%
    %%%%%%%%%%%%%%%%%%%%%%%%%%%%%%%%%%%%%%%%%%%%% FIGURE 3 B %%%%%%%%%%%%%%%%%%%%%%%%%%%%%%%%%%%%%%%%%%%%%%%%%%%%%%%%%%%%%%%%%%%%%%%%%%%%%%%%%%
    %%%%%%%%%%%%%%%%%%%%%%%%%%%%%%%%%%%%%%%%%%%%%%%%%%%%%%%%%%%%%%%%%%%%%%%%%%%%%%%%%%%%%%%%%%%%%%%%%%%%%%%%%%%%%%%%%%%%%%%%%%%%%%%%%%%%%%%%%%%

    \vspace{1cm}
	\hspace{-2cm}
	\begin{subfigure}[b]{0.6\linewidth}
	\resizebox{\linewidth}{!}{

\begin{tikzpicture}
    \node[draw=none, fill=none] at (-3.65,4.55) {\LARGE B};
        \draw[rotate=-45,very thick] (0,0) ellipse (4.5cm and 1.2cm);
        \draw [very thick,decorate,decoration={brace,mirror,amplitude=10pt},xshift=-4pt,yshift=0pt]
(3.5,-3.3) -- (3.5,3.3) node [black,midway,xshift=0.7cm] 
{\footnotesize \resizebox{0.5cm}{!}{$\bf \Sigma_1$}};
        \draw [very thick,decorate,decoration={brace,amplitude=10pt},xshift=-4pt,yshift=0pt]
(-3.3,3.5) -- (3.3,3.5) node [black,midway,yshift=0.7cm] 
{\footnotesize \resizebox{0.5cm}{!}{$\bf \Sigma_2$}};
 	\draw[very thick,-|] (0,0) - - (.8,.8)  node[rotate=45,pos=0.3,midway,fill=white] (stiff) {$w_1$} node[rotate=45,pos=0.3,midway,below] {\textbf{Stiff}};
	%\draw[very thick,->] (0,0) - - (.5,.5) node[rotate=45,pos=0.3,below] {\textbf{Stiff} \resizebox{0.8cm}{!}{$\bold (w_1)$}};
	%\draw[very thick,dashed,->] (0,0) - - (-2.5,2.5) node[rotate=-45,pos=0.3,below] {\textbf{Sloppy} \resizebox{0.8cm}{!}{$\bold (w_2)$}};
	\draw[very thick,-|] (0,0) - - (-3.1,3.1) node[rotate=-45,pos=0.3,midway,fill=white]{$w_2$} node[rotate=-45,pos=0.3,midway,below] {\textbf{Sloppy}};
	\draw[very thick, ->] (-3,-3) -- (-3,-2) node[pos=0.3,left] {$\log(\theta_1)$};
	\draw[very thick, ->] (-3,-3) -- (-2,-3) node[pos=0.3,below] {$\log(\theta_2)$};
	
\end{tikzpicture}
}

\end{subfigure}
\hspace{1cm}
\vspace{1cm}
    %%%%%%%%%%%%%%%%%%%%%%%%%%%%%%%%%%%%%%%%%%%%%%%%%%%%%%%%%%%%%%%%%%%%%%%%%%%%%%%%%%%%%%%%%%%%%%%%%%%%%%%%%%%%%%%%%%%%%%%%%%%%%%%%%%%%%%%%%%%
    %%%%%%%%%%%%%%%%%%%%%%%%%%%%%%%%%%%%%%%%%%%%% FIGURE 3 C %%%%%%%%%%%%%%%%%%%%%%%%%%%%%%%%%%%%%%%%%%%%%%%%%%%%%%%%%%%%%%%%%%%%%%%%%%%%%%%%%%
    %%%%%%%%%%%%%%%%%%%%%%%%%%%%%%%%%%%%%%%%%%%%%%%%%%%%%%%%%%%%%%%%%%%%%%%%%%%%%%%%%%%%%%%%%%%%%%%%%%%%%%%%%%%%%%%%%%%%%%%%%%%%%%%%%%%%%%%%%%%
\begin{subfigure}[b]{0.65\linewidth}
	\resizebox{\linewidth}{!}{
		\begin{tikzpicture}
            \node[draw=none, fill=none] at (-1.3,6.3) {\Large C};
			\begin{axis}[
			axis line style = { draw = none },
				ylabel = {$\log\left(1/{w_i^2}\right)$},
				xlabel = {Ellipsoid Axis Direction $\left(i: {\rm sloppy}\to{\rm stiff}\right)$},
				xtick pos=left,
				ytick={-10.6,-7.6,3.3,5.3},
				ytick pos=left,
				yticklabels={\textcolor{red}{-10.6},\textcolor{black}{-7.6},\textcolor{red}{3.3},\textcolor{black}{5.3}},
				xmajorticks=false,
				%yticklabels{}
			]
			\addplot[color=black,mark=*] table[x index=0, y index=1, col sep=comma, only marks] {../CSV/figure_3c_panel1_log10.csv};
			\addplot[color=red,mark=*] table[x index=0, y index=1, col sep=comma, only marks] {../CSV/figure_3c_panel1_orig.csv};
			\draw[scale=1,domain=0:8,dashed,variable=\x,black] plot ({\x},{-4.7+0*\x});
			%\draw[scale=1,domain=0:8,smooth,variable=\x,red] plot ({\x},{.8488*\x-5.2838});
			
			
			\end{axis}
			
	
		\end{tikzpicture}
	}
	\end{subfigure}

\hspace{-1cm}
\captionsetup{width=1.3\linewidth}
\caption{The BK model is sloppy.  (A) Lower bounds on parameter error (95\% confidence interval) for each of the $\po, \lpo$ assays.  The $\po$ assay exhibits much worse identifiability than the $\lpo$ assay.  (B). Ellipsoid of constant cost for a toy two-parameter model.  The parameters $\theta_{1,2}$ are constrained in the stiff direction, but have large error regions $\Sigma_{1,2}$ due to the presence of a large sloppy direction.  (C) Calculated $\log(1/{{\rm width}_i}^2$) values for the $\po$ assay (red) and $\lpo$ assay (black).  Both exhibit a linear trend, the signature of a sloppy model.}
\end{figure}



%%%%%%%%%%%%%%%%%%%%%%%%%%%%%%%%%%%%%%%%%%%%%%%%%%%%%%%%%%%%%%%%%%%%%%%%%%%%%%%%%%%%%%%%%%%%%%%%%%%%%%%%%%%%%%%%%%%%%%%%%%%%%%%%%%%%%%%%%%%%%%%
%%%%%%%%%%%%%%%%%%%%%%%%%%%%%%%%%%%%%%%%%%%%%%%%%%%%%%%%%%%%%%%%%%%%%%%%%%%%%%%%%%%%%%%%%%%%%%%%%%%%%%%%%%%%%%%%%%%%%%%%%%%%%%%%%%%%%%%%%%%%%%%
%%%%%%%%%%%%%%%%%%%%%%%%%%%%%%%%%%%%%%%%%%%%% FIGURE 4 %%%%%%%%%%%%%%%%%%%%%%%%%%%%%%%%%%%%%%%%%%%%%%%%%%%%%%%%%%%%%%%%%%%%%%%%%%%%%%%%%%%%%%%%
%%%%%%%%%%%%%%%%%%%%%%%%%%%%%%%%%%%%%%%%%%%%%%%%%%%%%%%%%%%%%%%%%%%%%%%%%%%%%%%%%%%%%%%%%%%%%%%%%%%%%%%%%%%%%%%%%%%%%%%%%%%%%%%%%%%%%%%%%%%%%%%
%%%%%%%%%%%%%%%%%%%%%%%%%%%%%%%%%%%%%%%%%%%%%%%%%%%%%%%%%%%%%%%%%%%%%%%%%%%%%%%%%%%%%%%%%%%%%%%%%%%%%%%%%%%%%%%%%%%%%%%%%%%%%%%%%%%%%%%%%%%%%%%

\begin{figure}


    %%%%%%%%%%%%%%%%%%%%%%%%%%%%%%%%%%%%%%%%%%%%%%%%%%%%%%%%%%%%%%%%%%%%%%%%%%%%%%%%%%%%%%%%%%%%%%%%%%%%%%%%%%%%%%%%%%%%%%%%%%%%%%%%%%%%%%%%%%%
    %%%%%%%%%%%%%%%%%%%%%%%%%%%%%%%%%%%%%%%%%%%%% FIGURE 4 A %%%%%%%%%%%%%%%%%%%%%%%%%%%%%%%%%%%%%%%%%%%%%%%%%%%%%%%%%%%%%%%%%%%%%%%%%%%%%%%%%%
    %%%%%%%%%%%%%%%%%%%%%%%%%%%%%%%%%%%%%%%%%%%%%%%%%%%%%%%%%%%%%%%%%%%%%%%%%%%%%%%%%%%%%%%%%%%%%%%%%%%%%%%%%%%%%%%%%%%%%%%%%%%%%%%%%%%%%%%%%%%


    %%%%%%%%%%%%%%%%%%%%%%%%%%%%%%%%%%%%%%%%%%%%%%%%%%%%%%%%%%%%%%%%%%%%%%%%%%%%%%%%%%%%%%%%%%%%%%%%%%%%%%%%%%%%%%%%%%%%%%%%%%%%%%%%%%%%%%%%%%%
    %%%%%%%%%%%%%%%%%%%%%%%%%%%%%%%%%%%%%%%%%%%%% FIGURE 4 B left%%%%%%%%%%%%%%%%%%%%%%%%%%%%%%%%%%%%%%%%%%%%%%%%%%%%%%%%%%%%%%%%%%%%%%%%%%%%%%
    %%%%%%%%%%%%%%%%%%%%%%%%%%%%%%%%%%%%%%%%%%%%%%%%%%%%%%%%%%%%%%%%%%%%%%%%%%%%%%%%%%%%%%%%%%%%%%%%%%%%%%%%%%%%%%%%%%%%%%%%%%%%%%%%%%%%%%%%%%%
\hspace{-1.5cm}
	\begin{subfigure}[b]{0.6\linewidth}
	\resizebox{\linewidth}{!}{

\begin{tikzpicture}
    \node[draw=none, fill=none] at (0, 10) {\Huge A};
    
    \node[draw=none,fill=none] (R) at (.75,3.5) {\Huge R};
    \node[draw=none,fill=none] (A) at (.75,0.5) {\Huge A};
    \draw [<- >, very thick] (A) to node[anchor=west] {\huge{${J}_0^{{(z_J)}}$}} (R);
    %\draw[<->] (A) to {v} (RA);% {V};
    %\draw[->, bend right=22.5] (\from) to node[fill=white] {$T_{\from \to}$}
    \draw [black, very thick] (0,0) to [square left brace ] (0,4);
    \draw [black, very thick] (1.5,0) to [square right brace] (1.5,4);
    
    \node[draw=none,fill=none,] (phione) at (.75,7.25) {\Huge $\phi_1 = \sqrt[4]{\Lo}D$};
    
%    \node[draw=none,fill=none,] (phione) at (10.75,7.25) {\Huge $\rcancel{\phi_2 = \sqrt[4]{\Lo}{C}}$};
    
 %   \node[draw=none,fill=none,] (phione) at (5.75,1) {\Huge $\rcancel{\phi_3 = \sqrt[4]{\Lo}{E}}$};
    
     \node[draw=none,fill=none,] (phione) at (6.9,3.75) {\Huge ${\phi_4 = {CE}}$};

    
    \draw [black, very thick] (10,0) to [square left brace ] (10,4);
    \draw [black, very thick] (11.5,0) to [square right brace] (11.5,4);
        \node[draw=none,fill=none] (X) at (10.75,3.5) {\Huge X};
    \node[draw=none,fill=none] (Xca) at (10.75,0.5) {{\huge{X}$\cdot\rm \large{Ca^{2+}}$}};
    \draw [<->, very thick, name=K] (X) to node[anchor=west] {\huge{${{K_d}}$}} (Xca);
    
    \draw [black, very thick] (5,5) to [square left brace ] (5,9);
    \draw [black, very thick] (6.5,5) to [square right brace] (6.5,9);
    
    \node[draw=none,fill=none] (C) at (5.75,8.5) {\Huge C};
    \node[draw=none,fill=none] (O) at (5.75,5.5) {\huge{O}};
    \draw [<->, very thick, name=L] (C) to node[anchor=west] {\huge $\rcancel[mediumcandyapplered]{{L}}_0^{{\rcancel[mediumcandyapplered]{(z_L)}}}$} (O);
        \node [draw=none,fill=none] (four) at (2.3,0) {\huge 4};
    
	    \node [draw=none,fill=none] (four) at (12.3,0) {\huge 4};
    
    %D should be at (2.5+.75,7) $5.75+2.5=8.25 10.75-2.5 = 
    
    %guide to locations:
    	%J_v node is at (1.8,2)
        %K node is at (10.8,2)
        %L node is at (6.25,6.8)
        %D node is at (2.5,6)
        %C node is at (9,6)
        %E node is at 5.75,2
    
    
    \node[draw=none,fill=none,rotate=45] (D) at (2.5,6) {\Huge $ {D} $};
    
    \node[draw=none,fill=none,rotate=45, left of=D, xshift=.2cm] (<D) at (2.5,6) {\Huge $<$};
    
     \node[draw=none,fill=none,rotate=45, right of=D, xshift=-.2cm] (D>) at (2.5,6) {\Huge $>$};    
     
         \node[draw=none,fill=none,rotate=-45] (C) at (9,6) {\Huge $   {C} $};

         
    \node[draw=none,fill=none,rotate=45, above of=C, yshift=-.2cm] (<C) at (9,6) {\Huge $\wedge$};
    
     \node[draw=none,fill=none,rotate=45, below of=C, yshift=.2cm] (C>) at (9,6) {\Huge $\vee$};    

    
    \node[draw=none,fill=none] (E) at (5.75,2) {\Huge $ {E} $};
    
    
        \node[draw=none,fill=none, left of=E, xshift=.2cm] (<E) at (5.75,2) {\Huge $<$};
    
     \node[draw=none,fill=none, right of=E, xshift=-.2cm] (E>) at (5.75,2) {\Huge $>$};   
              \draw[very thick,mediumcandyapplered,line width=1.25pt] (D.north west)--(D.south east);
        \draw[very thick,mediumcandyapplered,line width=1.25pt] (D.north east)--(D.south west); 
        
                      \draw[very thick,mediumcandyapplered,line width=1.25pt] (C.north west)--(C.south east);
        \draw[ very thick,mediumcandyapplered,line width=1.25pt] (C.north east)--(C.south west); 
        
                              \draw[very thick,mediumcandyapplered,line width=1.25pt] (E.north west)--(E.south east);
        \draw[very thick,mediumcandyapplered,line width=1.25pt] (E.north east)--(E.south west); 
    
    
 %   \draw [-, line width=0.7mm, red, bend left=15,dashed] (6.25,6.8) to node[anchor=east,font=\bf] {\huge$\phi_1=\sqrt[4]{L_0}E$} (5.75,2);
    
%    \draw [-, line width=0.7mm, red, dashed] (6.25,7.25) .. controls +(up:6cm) and +(left:1cm) .. node[above,sloped,font=\bf] {\huge$\mathbf \phi_2=\sqrt[4]{L_0}D$} (2.5,6);
    
 %       \draw [-, line width=0.7mm, red, dashed] (6.25,7.25) .. controls +(up:6cm) and +(right:1cm) .. node[above,sloped,font=\bf] {\huge$\phi_3=\sqrt[4]{L_0}C$} (9,6);
\end{tikzpicture}
}

\end{subfigure}
	\begin{subfigure}[b]{0.6\linewidth}


    %%%%%%%%%%%%%%%%%%%%%%%%%%%%%%%%%%%%%%%%%%%%%%%%%%%%%%%%%%%%%%%%%%%%%%%%%%%%%%%%%%%%%%%%%%%%%%%%%%%%%%%%%%%%%%%%%%%%%%%%%%%%%%%%%%%%%%%%%%%
    %%%%%%%%%%%%%%%%%%%%%%%%%%%%%%%%%%%%%%%%%%%%% FIGURE 4 B right %%%%%%%%%%%%%%%%%%%%%%%%%%%%%%%%%%%%%%%%%%%%%%%%%%%%%%%%%%%%%%%%%%%%%%%%%%%%
    %%%%%%%%%%%%%%%%%%%%%%%%%%%%%%%%%%%%%%%%%%%%%%%%%%%%%%%%%%%%%%%%%%%%%%%%%%%%%%%%%%%%%%%%%%%%%%%%%%%%%%%%%%%%%%%%%%%%%%%%%%%%%%%%%%%%%%%%%%%


	\resizebox{\linewidth}{!}{
\begin{tikzpicture}
\begin{axis}
       [
         axis line style = { draw = none },
         %ymode=log,
         xtick           = {-100,0,200}         ,
         %ytick           = {0,.5,1}         ,
         tick pos        = left           ,
         xlabel={V (mv)},
         ylabel = {$P_o$},
       ]

\pgfplotsset{cycle list set=0}
\foreach \i/\j in \CaSTR
{
\addplot +[only marks] table [x index=0, y index=1, col sep=comma] {../CSV/figure_1b_panel1_ca\i.csv};\label{\j}
}
\pgfplotsset{cycle list set=0}
\foreach \i/\j in \CaSTR
{
\addplot +[no markers] table [x index=0, y index=1, col sep=comma, mark=circle]{../CSV/figure_4b_panel2_ca\i.csv};
}

%\addplot table [x index=0, y index=0, col sep=comma] {figure_3b_panel1_paramzJ.csv};
%\addplot table [x=trila, y=st]
    %     \node [left] at (axis cs:  6,6) {$32hhhh$};
\end{axis}

\end{tikzpicture}
}

\end{subfigure}

\vspace{1cm}

    %%%%%%%%%%%%%%%%%%%%%%%%%%%%%%%%%%%%%%%%%%%%%%%%%%%%%%%%%%%%%%%%%%%%%%%%%%%%%%%%%%%%%%%%%%%%%%%%%%%%%%%%%%%%%%%%%%%%%%%%%%%%%%%%%%%%%%%%%%%
    %%%%%%%%%%%%%%%%%%%%%%%%%%%%%%%%%%%%%%%%%%%%% FIGURE 4 C left %%%%%%%%%%%%%%%%%%%%%%%%%%%%%%%%%%%%%%%%%%%%%%%%%%%%%%%%%%%%%%%%%%%%%%%%%%%%%
    %%%%%%%%%%%%%%%%%%%%%%%%%%%%%%%%%%%%%%%%%%%%%%%%%%%%%%%%%%%%%%%%%%%%%%%%%%%%%%%%%%%%%%%%%%%%%%%%%%%%%%%%%%%%%%%%%%%%%%%%%%%%%%%%%%%%%%%%%%%
\hspace{-1.5cm}
	\begin{subfigure}[b]{0.6\linewidth}
	\resizebox{\linewidth}{!}{

\begin{tikzpicture}
    \node[draw=none, fill=none] at (0, 10) {\Huge B};
    
    \node[draw=none,fill=none] (R) at (.75,3.5) {\Huge R};
    \node[draw=none,fill=none] (A) at (.75,0.5) {\Huge A};
    \draw [<- >, very thick] (A) to node[anchor=west] {\huge{$\rcancel{J}_0^{{(z_J)}}$}} (R);
    %\draw[<->] (A) to {v} (RA);% {V};
    %\draw[->, bend right=22.5] (\from) to node[fill=white] {$T_{\from \to}$}
    \draw [black, very thick] (0,0) to [square left brace ] (0,4);
    \draw [black, very thick] (1.5,0) to [square right brace] (1.5,4);
    
    \node[draw=none,fill=none,] (phione) at (.75,7.25) {\Huge $\phi_5 = J_0\sqrt[4]{\Lo}D$};
    
%    \node[draw=none,fill=none,] (phione) at (10.75,7.25) {\Huge $\rcancel{\phi_2 = \sqrt[4]{\Lo}{C}}$};
    
 %   \node[draw=none,fill=none,] (phione) at (5.75,1) {\Huge $\rcancel{\phi_3 = \sqrt[4]{\Lo}{E}}$};
    
     \node[draw=none,fill=none,] (phione) at (6.9,3.75) {\Huge ${\phi_4 = {CE}}$};

    
    \draw [black, very thick] (10,0) to [square left brace ] (10,4);
    \draw [black, very thick] (11.5,0) to [square right brace] (11.5,4);
        \node[draw=none,fill=none] (X) at (10.75,3.5) {\Huge X};
    \node[draw=none,fill=none] (Xca) at (10.75,0.5) {{\huge{X}$\cdot\rm \large{Ca^{2+}}$}};
    \draw [<->, very thick, name=K] (X) to node[anchor=west] {\huge{${{K_d}}$}} (Xca);
    
    \draw [black, very thick] (5,5) to [square left brace ] (5,9);
    \draw [black, very thick] (6.5,5) to [square right brace] (6.5,9);
    
    \node[draw=none,fill=none] (C) at (5.75,8.5) {\Huge C};
    \node[draw=none,fill=none] (O) at (5.75,5.5) {\huge{O}};
    \draw [<->, very thick, name=L] (C) to node[anchor=west] {\huge $\rcancel[mediumcandyapplered]{{L}}_0^{{\rcancel[mediumcandyapplered]{(z_L)}}}$} (O);
        \node [draw=none,fill=none] (four) at (2.3,0) {\huge 4};
    
	    \node [draw=none,fill=none] (four) at (12.3,0) {\huge 4};
    
    %D should be at (2.5+.75,7) $5.75+2.5=8.25 10.75-2.5 = 
    
    %guide to locations:
    	%J_v node is at (1.8,2)
        %K node is at (10.8,2)
        %L node is at (6.25,6.8)
        %D node is at (2.5,6)
        %C node is at (9,6)
        %E node is at 5.75,2
    
    
    \node[draw=none,fill=none,rotate=45] (D) at (2.5,6) {\Huge $ {D} $};
    
    \node[draw=none,fill=none,rotate=45, left of=D, xshift=.2cm] (<D) at (2.5,6) {\Huge $<$};
    
     \node[draw=none,fill=none,rotate=45, right of=D, xshift=-.2cm] (D>) at (2.5,6) {\Huge $>$};    
     
         \node[draw=none,fill=none,rotate=-45] (C) at (9,6) {\Huge $   {C} $};

         
    \node[draw=none,fill=none,rotate=45, above of=C, yshift=-.2cm] (<C) at (9,6) {\Huge $\wedge$};
    
     \node[draw=none,fill=none,rotate=45, below of=C, yshift=.2cm] (C>) at (9,6) {\Huge $\vee$};    

    
    \node[draw=none,fill=none] (E) at (5.75,2) {\Huge $ {E} $};
    
    
        \node[draw=none,fill=none, left of=E, xshift=.2cm] (<E) at (5.75,2) {\Huge $<$};
    
     \node[draw=none,fill=none, right of=E, xshift=-.2cm] (E>) at (5.75,2) {\Huge $>$};   
              \draw[very thick,mediumcandyapplered,line width=1.25pt] (D.north west)--(D.south east);
        \draw[very thick,mediumcandyapplered,line width=1.25pt] (D.north east)--(D.south west); 
        
                      \draw[very thick,mediumcandyapplered,line width=1.25pt] (C.north west)--(C.south east);
        \draw[ very thick,mediumcandyapplered,line width=1.25pt] (C.north east)--(C.south west); 
        
                              \draw[very thick,mediumcandyapplered,line width=1.25pt] (E.north west)--(E.south east);
        \draw[very thick,mediumcandyapplered,line width=1.25pt] (E.north east)--(E.south west); 
    
    
 %   \draw [-, line width=0.7mm, red, bend left=15,dashed] (6.25,6.8) to node[anchor=east,font=\bf] {\huge$\phi_1=\sqrt[4]{L_0}E$} (5.75,2);
    
%    \draw [-, line width=0.7mm, red, dashed] (6.25,7.25) .. controls +(up:6cm) and +(left:1cm) .. node[above,sloped,font=\bf] {\huge$\mathbf \phi_2=\sqrt[4]{L_0}D$} (2.5,6);
    
 %       \draw [-, line width=0.7mm, red, dashed] (6.25,7.25) .. controls +(up:6cm) and +(right:1cm) .. node[above,sloped,font=\bf] {\huge$\phi_3=\sqrt[4]{L_0}C$} (9,6);
\end{tikzpicture}
}

\end{subfigure}
    %%%%%%%%%%%%%%%%%%%%%%%%%%%%%%%%%%%%%%%%%%%%%%%%%%%%%%%%%%%%%%%%%%%%%%%%%%%%%%%%%%%%%%%%%%%%%%%%%%%%%%%%%%%%%%%%%%%%%%%%%%%%%%%%%%%%%%%%%%%
    %%%%%%%%%%%%%%%%%%%%%%%%%%%%%%%%%%%%%%%%%%%%% FIGURE 4 C right %%%%%%%%%%%%%%%%%%%%%%%%%%%%%%%%%%%%%%%%%%%%%%%%%%%%%%%%%%%%%%%%%%%%%%%%%%%%
    %%%%%%%%%%%%%%%%%%%%%%%%%%%%%%%%%%%%%%%%%%%%%%%%%%%%%%%%%%%%%%%%%%%%%%%%%%%%%%%%%%%%%%%%%%%%%%%%%%%%%%%%%%%%%%%%%%%%%%%%%%%%%%%%%%%%%%%%%%%
	\begin{subfigure}[b]{0.6\linewidth}

	\resizebox{\linewidth}{!}{
\begin{tikzpicture}
\begin{axis}
       [
         axis line style = { draw = none },
         %ymode=log,
         xtick           = {-100,0,200}         ,
         %ytick           = {0,.5,1}         ,
         tick pos        = left           ,
         xlabel={V (mv)},
         ylabel = {$P_o$},
       ]
       
       
\pgfplotsset{cycle list set=0}
\foreach \i/\j in \CaSTR
{
\addplot +[only marks] table [x index=0, y index=1, col sep=comma] {../CSV/figure_1b_panel1_ca\i.csv};
}
\pgfplotsset{cycle list set=0}
\foreach \i/\j in \CaSTR
{
\addplot +[no markers] table [x index=0, y index=1, col sep=comma, mark=circle]{../CSV/figure_4c_panel2_ca\i.csv};
}

%\addplot table [x index=0, y index=0, col sep=comma] {figure_3b_panel1_paramzJ.csv};
%\addplot table [x=trila, y=st]
    %     \node [left] at (axis cs:  6,6) {$32hhhh$};
\end{axis}

\end{tikzpicture}
}
\end{subfigure}

\captionsetup{width=1.3\linewidth}
\caption{Results of model reduction for $\po.$ (A, left) Schematic of the model admitted by the third reduction; five parameters have been eliminated from the original model (red X) and two new `emergent' parameters have been added ($\phi_{1,2}$) for a total reduction of three parameters.  (A, right). This model fits the data (solid lines) extremely well.  (B, left). Schematic of model admitted by the fourth reduction.  This model does not fit the data well at low $\ca$ (B, right).  Synthetic data labeled as previously.}

\end{figure}

\begin{figure}
\begin{center}
	\begin{subfigure}[b]{0.48\linewidth}
	\centering
	\resizebox{\linewidth}{!}{

\begin{tikzpicture}
\begin{axis}
       [
         axis line style = { draw = none },
         %ymode=log,
        % xtick           = {-100,0,160}         ,
         %ytick           = {0,.5,1}         ,
         tick pos        = left           ,
         xlabel={reduction step},
         ylabel = {RMS error},
       ]

\pgfplotsset{cycle list set=0}
\addplot table [x index=0, y index=1, col sep=comma] {../CSV/figure_4a_panel1.csv};

\end{axis}

\end{tikzpicture}
}

\end{subfigure}
\end{center}
\captionsetup{labelformat=empty}
\caption{Figure 4 - supplement figure 1: RMS cost for each reduced model.  Note a small uptick in error for the fourth reduction, consistent with the poor fit to low $[\ca]$ in Figure 4B.  Subsequent reductions do not fit the data well.}
\end{figure}

\addtocounter{figure}{-1}

\pagebreak

%%%%%%%%%%%%%%%%%%%%%%%%%%%%%%%%%%%%%%%%%%%%%%%%%%%%%%%%%%%%%%%%%%%%%%%%%%%%%%%%%%%%%%%%%%%%%%%%%%%%%%%%%%%%%%%%%%%%%%%%%%%%%%%%%%%%%%%%%%%%%%%
%%%%%%%%%%%%%%%%%%%%%%%%%%%%%%%%%%%%%%%%%%%%%%%%%%%%%%%%%%%%%%%%%%%%%%%%%%%%%%%%%%%%%%%%%%%%%%%%%%%%%%%%%%%%%%%%%%%%%%%%%%%%%%%%%%%%%%%%%%%%%%%
%%%%%%%%%%%%%%%%%%%%%%%%%%%%%%%%%%%%%%%%%%%%% FIGURE 5 %%%%%%%%%%%%%%%%%%%%%%%%%%%%%%%%%%%%%%%%%%%%%%%%%%%%%%%%%%%%%%%%%%%%%%%%%%%%%%%%%%%%%%%%
%%%%%%%%%%%%%%%%%%%%%%%%%%%%%%%%%%%%%%%%%%%%%%%%%%%%%%%%%%%%%%%%%%%%%%%%%%%%%%%%%%%%%%%%%%%%%%%%%%%%%%%%%%%%%%%%%%%%%%%%%%%%%%%%%%%%%%%%%%%%%%%
%%%%%%%%%%%%%%%%%%%%%%%%%%%%%%%%%%%%%%%%%%%%%%%%%%%%%%%%%%%%%%%%%%%%%%%%%%%%%%%%%%%%%%%%%%%%%%%%%%%%%%%%%%%%%%%%%%%%%%%%%%%%%%%%%%%%%%%%%%%%%%%

\begin{figure}


    %%%%%%%%%%%%%%%%%%%%%%%%%%%%%%%%%%%%%%%%%%%%%%%%%%%%%%%%%%%%%%%%%%%%%%%%%%%%%%%%%%%%%%%%%%%%%%%%%%%%%%%%%%%%%%%%%%%%%%%%%%%%%%%%%%%%%%%%%%%
    %%%%%%%%%%%%%%%%%%%%%%%%%%%%%%%%%%%%%%%%%%%%% FIGURE 5 A %%%%%%%%%%%%%%%%%%%%%%%%%%%%%%%%%%%%%%%%%%%%%%%%%%%%%%%%%%%%%%%%%%%%%%%%%%%%%%%%%%
    %%%%%%%%%%%%%%%%%%%%%%%%%%%%%%%%%%%%%%%%%%%%%%%%%%%%%%%%%%%%%%%%%%%%%%%%%%%%%%%%%%%%%%%%%%%%%%%%%%%%%%%%%%%%%%%%%%%%%%%%%%%%%%%%%%%%%%%%%%%




\vspace{1cm}
\hspace{-.5cm}
		\begin{subfigure}[b]{0.48\linewidth}
	\centering
	\resizebox{\linewidth}{!}{
\begin{tikzpicture}
\node[draw=none, fill=none] at (0.3,5.8) {\Large A};
\begin{axis}
       [
         axis line style = { draw = none },
         xtick = \empty,
         ytick = \empty,
         %xtick           = {-10,6}         ,
         %ytick           = {-.544727175441672, -13.027053197600004, -10.151948911834628, .6931471805599453,
    %     -.8675005677047232}         ,
         tick pos        = left           ,
                  xmax = 6.14,
       %  yticklabels={$z_J$, $L_0$, $K_d$, $E$, $z_L$},
         %ylabel={$\log(\theta)$},
        % xlabel = {time},
       ]

\foreach \i in {L0, J0, zJ, Kd, C, D, E}
{
\addplot[no markers, very thick, color=black] table [x index=0, y index=1, col sep=comma] {../CSV/figure_5a_panel1_param_\i.csv};
}
\foreach \i in {zL}
{
\addplot[no markers, very thick, color=red] table [x index=0, y index=1, col sep=comma] {../CSV/figure_5a_panel1_param_\i.csv};
}


\node[pos=0.0, pin=left:``first point'']{} ;




%\addplot table [x=trila, y=st]
        % \node [left] at (axis cs:  6,6) {$32hhhh$};
\end{axis}

\end{tikzpicture}
}
	\end{subfigure}
	\hspace{1cm}
	\begin{subfigure}[b]{0.6\linewidth}
	\centering
	\resizebox{\linewidth}{!}{

\begin{tikzpicture}
    \node[draw=none,fill=none] (R) at (.75,3.5) {\Huge R};
    \node[draw=none,fill=none] (A) at (.75,0.5) {\Huge A};
    \draw [<- >, very thick] (A) to node[anchor=west] {\huge{$J_0^{(z_J)}$}} (R);
    %\draw[<->] (A) to {v} (RA);% {V};
    %\draw[->, bend right=22.5] (\from) to node[fill=white] {$T_{\from \to}$}
    \draw [black, very thick] (0,0) to [square left brace ] (0,4);
    \draw [black, very thick] (1.5,0) to [square right brace] (1.5,4);

    
    \draw [black, very thick] (10,0) to [square left brace ] (10,4);
    \draw [black, very thick] (11.5,0) to [square right brace] (11.5,4);
        \node[draw=none,fill=none] (X) at (10.75,3.5) {\Huge X};
    \node[draw=none,fill=none] (Xca) at (10.75,0.5) {{\huge{X}$\cdot\rm \large{Ca^{2+}}$}};
    \draw [<->, very thick, name=K] (X) to node[anchor=west] {\huge{$K_d$}} (Xca);
    
    \draw [black, very thick] (5,5) to [square left brace ] (5,9);
    \draw [black, very thick] (6.5,5) to [square right brace] (6.5,9);
    
    \node[draw=none,fill=none] (C) at (5.75,8.5) {\Huge C};
    \node[draw=none,fill=none] (O) at (5.75,5.5) {\huge{O}};
    \draw [<->, very thick, name=L] (C) to node[anchor=west] {\huge ${L}_0^{{{(\rcancel{z_L})}}}$} (O);
        \node [draw=none,fill=none] (four) at (2.3,0) {\huge 4};
    
	    \node [draw=none,fill=none] (four) at (12.3,0) {\huge 4};
    
    %D should be at (2.5+.75,7) $5.75+2.5=8.25 10.75-2.5 = 
    
    %guide to locations:
    	%J_v node is at (1.8,2)
        %K node is at (10.8,2)
        %L node is at (6.25,6.8)
        %D node is at (2.5,6)
        %C node is at (9,6)
        %E node is at 5.75,2
    
    
    \node[draw=none,fill=none,rotate=45] (D) at (2.5,6) {\Huge $ {D} $};
    
    \node[draw=none,fill=none,rotate=45, left of=D, xshift=.2cm] (<D) at (2.5,6) {\Huge $<$};
    
     \node[draw=none,fill=none,rotate=45, right of=D, xshift=-.2cm] (D>) at (2.5,6) {\Huge $>$};    
     
         \node[draw=none,fill=none,rotate=-45] (C) at (9,6) {\Huge $   {C} $};

         
    \node[draw=none,fill=none,rotate=45, above of=C, yshift=-.2cm] (<C) at (9,6) {\Huge $\wedge$};
    
     \node[draw=none,fill=none,rotate=45, below of=C, yshift=.2cm] (C>) at (9,6) {\Huge $\vee$};    

    
    \node[draw=none,fill=none] (E) at (5.75,2) {\Huge $ {E} $};
    
    
        \node[draw=none,fill=none, left of=E, xshift=.2cm] (<E) at (5.75,2) {\Huge $<$};
    
     \node[draw=none,fill=none, right of=E, xshift=-.2cm] (E>) at (5.75,2) {\Huge $>$};    
    
 %   \draw [-, line width=0.7mm, red, bend left=15,dashed] (6.25,6.8) to node[anchor=east,font=\bf] {\huge$\phi_1=\sqrt[4]{L_0}E$} (5.75,2);
    
%    \draw [-, line width=0.7mm, red, dashed] (6.25,7.25) .. controls +(up:6cm) and +(left:1cm) .. node[above,sloped,font=\bf] {\huge$\mathbf \phi_2=\sqrt[4]{L_0}D$} (2.5,6);
    
 %       \draw [-, line width=0.7mm, red, dashed] (6.25,7.25) .. controls +(up:6cm) and +(right:1cm) .. node[above,sloped,font=\bf] {\huge$\phi_3=\sqrt[4]{L_0}C$} (9,6);
\end{tikzpicture}
}

\end{subfigure}
    
    %%%%%%%%%%%%%%%%%%%%%%%%%%%%%%%%%%%%%%%%%%%%%%%%%%%%%%%%%%%%%%%%%%%%%%%%%%%%%%%%%%%%%%%%%%%%%%%%%%%%%%%%%%%%%%%%%%%%%%%%%%%%%%%%%%%%%%%%%%%
    %%%%%%%%%%%%%%%%%%%%%%%%%%%%%%%%%%%%%%%%%%%%% FIGURE 5 B %%%%%%%%%%%%%%%%%%%%%%%%%%%%%%%%%%%%%%%%%%%%%%%%%%%%%%%%%%%%%%%%%%%%%%%%%%%%%%%%%%
    %%%%%%%%%%%%%%%%%%%%%%%%%%%%%%%%%%%%%%%%%%%%%%%%%%%%%%%%%%%%%%%%%%%%%%%%%%%%%%%%%%%%%%%%%%%%%%%%%%%%%%%%%%%%%%%%%%%%%%%%%%%%%%%%%%%%%%%%%%%
\vspace{1cm}
\hspace{-.5cm}
	\begin{subfigure}[b]{0.48\linewidth}
	\centering
	\resizebox{\linewidth}{!}{
	\begin{tikzpicture}
    \node[draw=none, fill=none] at (0.3,5.8) {\Large B};

	\begin{axis}
       [
         axis line style = { draw = none },
         xtick           = {1,1}         ,
         ytick           = \empty         ,
                  xmax = 6.14,
         tick pos        = left           ,
        % ylabel={$\log(\theta)$},
	xtick=\empty,
         %xlabel = {time},
       ]

\foreach \i in {J0, zJ, Kd}
{
\addplot[no markers, very thick, color=black] table [x index=0, y index=1, col sep=comma] {../CSV/figure_5b_panel1_param_\i.csv};
}
\foreach \i in {L0, C, D, E}
{
\addplot[no markers, very thick, color=red] table [x index=0, y index=1, col sep=comma] {../CSV/figure_5b_panel1_param_\i.csv};
}
%\node [left] at (a) {bbb};

%\node [right] at (axis cs: 4.06654921543317,-10.158929504340358) {bbb};



%\addplot table [x=trila, y=st]
        % \node [left] at (axis cs:  6,6) {$32hhhh$};
\end{axis}

\end{tikzpicture}
}

	\end{subfigure}
		\hspace{1cm}
	\begin{subfigure}[b]{0.6\linewidth}
	\resizebox{\linewidth}{!}{

\begin{tikzpicture}
    \node[draw=none,fill=none] (R) at (.75,3.5) {\Huge R};
    \node[draw=none,fill=none] (A) at (.75,0.5) {\Huge A};
    \draw [<- >, very thick] (A) to node[anchor=west] {\huge{${J}_0^{{(z_J)}}$}} (R);
    %\draw[<->] (A) to {v} (RA);% {V};
    %\draw[->, bend right=22.5] (\from) to node[fill=white] {$T_{\from \to}$}
    \draw [black, very thick] (0,0) to [square left brace ] (0,4);
    \draw [black, very thick] (1.5,0) to [square right brace] (1.5,4);
    
    \node[draw=none,fill=none,] (phione) at (.75,7.25) {\Huge $\phi_1 = \sqrt[4]{\Lo}D$};
    
    \node[draw=none,fill=none,] (phione) at (10.75,7.25) {\Huge $\phi_2 = \sqrt[4]{\Lo}{C}$};
    
    \node[draw=none,fill=none,] (phione) at (5.75,1) {\Huge $\phi_3 = \left.{E}\middle/{\sqrt[4]{\Lo}}\right.$};

    
    \draw [black, very thick] (10,0) to [square left brace ] (10,4);
    \draw [black, very thick] (11.5,0) to [square right brace] (11.5,4);
        \node[draw=none,fill=none] (X) at (10.75,3.5) {\Huge X};
    \node[draw=none,fill=none] (Xca) at (10.75,0.5) {{\huge{X}$\cdot\rm \large{Ca^{2+}}$}};
    \draw [<->, very thick, name=K] (X) to node[anchor=west] {\huge{${{K_d}}$}} (Xca);
    
    \draw [black, very thick] (5,5) to [square left brace ] (5,9);
    \draw [black, very thick] (6.5,5) to [square right brace] (6.5,9);
    
    \node[draw=none,fill=none] (C) at (5.75,8.5) {\Huge C};
    \node[draw=none,fill=none] (O) at (5.75,5.5) {\huge{O}};
    \draw [<->, very thick, name=L] (C) to node[anchor=west] {\huge $\rcancel[mediumcandyapplered]{{L}}_0^{{\rcancel[mediumcandyapplered]{(z_L)}}}$} (O);
        \node [draw=none,fill=none] (four) at (2.3,0) {\huge 4};
    
	    \node [draw=none,fill=none] (four) at (12.3,0) {\huge 4};
    
    %D should be at (2.5+.75,7) $5.75+2.5=8.25 10.75-2.5 = 
    
    %guide to locations:
    	%J_v node is at (1.8,2)
        %K node is at (10.8,2)
        %L node is at (6.25,6.8)
        %D node is at (2.5,6)
        %C node is at (9,6)
        %E node is at 5.75,2
    
    
    \node[draw=none,fill=none,rotate=45] (D) at (2.5,6) {\Huge $ {D} $};
    
    \node[draw=none,fill=none,rotate=45, left of=D, xshift=.2cm] (<D) at (2.5,6) {\Huge $<$};
    
     \node[draw=none,fill=none,rotate=45, right of=D, xshift=-.2cm] (D>) at (2.5,6) {\Huge $>$};    
     
         \node[draw=none,fill=none,rotate=-45] (C) at (9,6) {\Huge $   {C} $};

         
    \node[draw=none,fill=none,rotate=45, above of=C, yshift=-.2cm] (<C) at (9,6) {\Huge $\wedge$};
    
     \node[draw=none,fill=none,rotate=45, below of=C, yshift=.2cm] (C>) at (9,6) {\Huge $\vee$};    

    
    \node[draw=none,fill=none] (E) at (5.75,2) {\Huge $ {E} $};
    
    
        \node[draw=none,fill=none, left of=E, xshift=.2cm] (<E) at (5.75,2) {\Huge $<$};
    
     \node[draw=none,fill=none, right of=E, xshift=-.2cm] (E>) at (5.75,2) {\Huge $>$};   
              \draw[very thick,mediumcandyapplered,line width=1.25pt] (D.north west)--(D.south east);
        \draw[very thick,mediumcandyapplered,line width=1.25pt] (D.north east)--(D.south west); 
        
                      \draw[very thick,mediumcandyapplered,line width=1.25pt] (C.north west)--(C.south east);
        \draw[ very thick,mediumcandyapplered,line width=1.25pt] (C.north east)--(C.south west); 
        
                              \draw[very thick,mediumcandyapplered,line width=1.25pt] (E.north west)--(E.south east);
        \draw[very thick,mediumcandyapplered,line width=1.25pt] (E.north east)--(E.south west); 
    
    
 %   \draw [-, line width=0.7mm, red, bend left=15,dashed] (6.25,6.8) to node[anchor=east,font=\bf] {\huge$\phi_1=\sqrt[4]{L_0}E$} (5.75,2);
    
%    \draw [-, line width=0.7mm, red, dashed] (6.25,7.25) .. controls +(up:6cm) and +(left:1cm) .. node[above,sloped,font=\bf] {\huge$\mathbf \phi_2=\sqrt[4]{L_0}D$} (2.5,6);
    
 %       \draw [-, line width=0.7mm, red, dashed] (6.25,7.25) .. controls +(up:6cm) and +(right:1cm) .. node[above,sloped,font=\bf] {\huge$\phi_3=\sqrt[4]{L_0}C$} (9,6);
\end{tikzpicture}
}

\end{subfigure}

\vspace{1cm}
    %%%%%%%%%%%%%%%%%%%%%%%%%%%%%%%%%%%%%%%%%%%%%%%%%%%%%%%%%%%%%%%%%%%%%%%%%%%%%%%%%%%%%%%%%%%%%%%%%%%%%%%%%%%%%%%%%%%%%%%%%%%%%%%%%%%%%%%%%%%
    %%%%%%%%%%%%%%%%%%%%%%%%%%%%%%%%%%%%%%%%%%%%% FIGURE 5 C %%%%%%%%%%%%%%%%%%%%%%%%%%%%%%%%%%%%%%%%%%%%%%%%%%%%%%%%%%%%%%%%%%%%%%%%%%%%%%%%%%
    %%%%%%%%%%%%%%%%%%%%%%%%%%%%%%%%%%%%%%%%%%%%%%%%%%%%%%%%%%%%%%%%%%%%%%%%%%%%%%%%%%%%%%%%%%%%%%%%%%%%%%%%%%%%%%%%%%%%%%%%%%%%%%%%%%%%%%%%%%%
\hspace{-.5cm}
	\begin{subfigure}[b]{0.48\linewidth}
		\resizebox{\linewidth}{!}{
	\begin{tikzpicture}
    \node[draw=none, fill=none] at (0.3,5.8) {\Large C};

	\begin{axis}
       [
         axis line style = { draw = none },
         xtick           = {1,1}         ,
         ytick           = \empty        ,
                  xmax = 6.14,
         tick pos        = left           ,
        % ylabel={$\log(\theta)$},
	xtick=\empty,
       %  xlabel = {time},
       ]


\foreach \i in {J0, zJ, Kd, LD}
{
\addplot[no markers, very thick, color=black] table [x index=0, y index=1, col sep=comma] {../CSV/figure_5c_panel1_param_\i.csv};
}
\foreach \i in {LC, EL}
{
\addplot[no markers, very thick, color=red] table [x index=0, y index=1, col sep=comma] {../CSV/figure_5c_panel1_param_\i.csv};
}
%\node [left] at (a) {bbb};


\draw[very thick, ->] (axis cs: 3,-10) -- (axis cs: 4.5,-10) node[pos=0.3,above] {\ \ time};
\draw[very thick, ->] (axis cs: 3,-10) -- (3,-5) node[pos=0.3,left] {$\log(\theta)$};



%\addplot table [x=trila, y=st]
        % \node [left] at (axis cs:  6,6) {$32hhhh$};
\end{axis}

\end{tikzpicture}
}

	\end{subfigure}
		\hspace{1cm}
	\begin{subfigure}[b]{0.6\linewidth}
	\resizebox{\linewidth}{!}{

\begin{tikzpicture}
    \node[draw=none,fill=none] (R) at (.75,3.5) {\Huge R};
    \node[draw=none,fill=none] (A) at (.75,0.5) {\Huge A};
    \draw [<- >, very thick] (A) to node[anchor=west] {\huge{${J}_0^{{(z_J)}}$}} (R);
    %\draw[<->] (A) to {v} (RA);% {V};
    %\draw[->, bend right=22.5] (\from) to node[fill=white] {$T_{\from \to}$}
    \draw [black, very thick] (0,0) to [square left brace ] (0,4);
    \draw [black, very thick] (1.5,0) to [square right brace] (1.5,4);
    
    \node[draw=none,fill=none,] (phione) at (.75,7.25) {\Huge $\phi_1 = \sqrt[4]{\Lo}D$};
    
    \node[draw=none,fill=none,] (phione) at (10.75,7.25) {\Huge $\rcancel{\phi_2 = \sqrt[4]{\Lo}{C}}$};
    
    \node[draw=none,fill=none,] (phione) at (5.75,1) {\Huge $\rcancel{\phi_3 = \left.{E}\middle/{\sqrt[4]{\Lo}}\right.}$};
    
     \node[draw=none,fill=none,] (phione) at (6.9,3.75) {\Huge ${\phi_4 = {CE}}$};

    
    \draw [black, very thick] (10,0) to [square left brace ] (10,4);
    \draw [black, very thick] (11.5,0) to [square right brace] (11.5,4);
        \node[draw=none,fill=none] (X) at (10.75,3.5) {\Huge X};
    \node[draw=none,fill=none] (Xca) at (10.75,0.5) {{\huge{X}$\cdot\rm \large{Ca^{2+}}$}};
    \draw [<->, very thick, name=K] (X) to node[anchor=west] {\huge{${{K_d}}$}} (Xca);
    
    \draw [black, very thick] (5,5) to [square left brace ] (5,9);
    \draw [black, very thick] (6.5,5) to [square right brace] (6.5,9);
    
    \node[draw=none,fill=none] (C) at (5.75,8.5) {\Huge C};
    \node[draw=none,fill=none] (O) at (5.75,5.5) {\huge{O}};
    \draw [<->, very thick, name=L] (C) to node[anchor=west] {\huge $\rcancel[mediumcandyapplered]{{L}}_0^{{\rcancel[mediumcandyapplered]{(z_L)}}}$} (O);
        \node [draw=none,fill=none] (four) at (2.3,0) {\huge 4};
    
	    \node [draw=none,fill=none] (four) at (12.3,0) {\huge 4};
    
    %D should be at (2.5+.75,7) $5.75+2.5=8.25 10.75-2.5 = 
    
    %guide to locations:
    	%J_v node is at (1.8,2)
        %K node is at (10.8,2)
        %L node is at (6.25,6.8)
        %D node is at (2.5,6)
        %C node is at (9,6)
        %E node is at 5.75,2
    
    
    \node[draw=none,fill=none,rotate=45] (D) at (2.5,6) {\Huge $ {D} $};
    
    \node[draw=none,fill=none,rotate=45, left of=D, xshift=.2cm] (<D) at (2.5,6) {\Huge $<$};
    
     \node[draw=none,fill=none,rotate=45, right of=D, xshift=-.2cm] (D>) at (2.5,6) {\Huge $>$};    
     
         \node[draw=none,fill=none,rotate=-45] (C) at (9,6) {\Huge $   {C} $};

         
    \node[draw=none,fill=none,rotate=45, above of=C, yshift=-.2cm] (<C) at (9,6) {\Huge $\wedge$};
    
     \node[draw=none,fill=none,rotate=45, below of=C, yshift=.2cm] (C>) at (9,6) {\Huge $\vee$};    

    
    \node[draw=none,fill=none] (E) at (5.75,2) {\Huge $ {E} $};
    
    
        \node[draw=none,fill=none, left of=E, xshift=.2cm] (<E) at (5.75,2) {\Huge $<$};
    
     \node[draw=none,fill=none, right of=E, xshift=-.2cm] (E>) at (5.75,2) {\Huge $>$};   
              \draw[very thick,mediumcandyapplered,line width=1.25pt] (D.north west)--(D.south east);
        \draw[very thick,mediumcandyapplered,line width=1.25pt] (D.north east)--(D.south west); 
        
                      \draw[very thick,mediumcandyapplered,line width=1.25pt] (C.north west)--(C.south east);
        \draw[ very thick,mediumcandyapplered,line width=1.25pt] (C.north east)--(C.south west); 
        
                              \draw[very thick,mediumcandyapplered,line width=1.25pt] (E.north west)--(E.south east);
        \draw[very thick,mediumcandyapplered,line width=1.25pt] (E.north east)--(E.south west); 
    
    
 %   \draw [-, line width=0.7mm, red, bend left=15,dashed] (6.25,6.8) to node[anchor=east,font=\bf] {\huge$\phi_1=\sqrt[4]{L_0}E$} (5.75,2);
    
%    \draw [-, line width=0.7mm, red, dashed] (6.25,7.25) .. controls +(up:6cm) and +(left:1cm) .. node[above,sloped,font=\bf] {\huge$\mathbf \phi_2=\sqrt[4]{L_0}D$} (2.5,6);
    
 %       \draw [-, line width=0.7mm, red, dashed] (6.25,7.25) .. controls +(up:6cm) and +(right:1cm) .. node[above,sloped,font=\bf] {\huge$\phi_3=\sqrt[4]{L_0}C$} (9,6);
\end{tikzpicture}
}

\end{subfigure}

\captionsetup{width=1.3\linewidth}
\caption{Intermediate MBAM steps, $\po$ assay.  The figure should be read left to right, top to bottom.  The left column displays the parameter values for a given model as MBAM progresses.  The reduced model created upon completion of the parameter search is displayed on the right.  (A). MBAM run for the full, original model.  There are eight lines, corresponding to eight parameters.  One of the parameters goes to zero; this is $z_L,$ and it is eliminated, giving our first reduced model, at right.  (B). In the second iteration, four parameters are observed to diverge: $L_0, \ D, \ E,\ C.$  These parameters are eliminated, and three new, emergent parameters are created ($\phi_{1,2,3}$), yielding a net reduction of one parameter.  (C). Two parameters are observed to diverge: $\phi_2, \ \phi_3$.  Note that there are only six lines, corresponding to the six remaining parameters.  The resulting model (right) has five parameters, and fits the data well (Figure 4).}




\end{figure}


%%%%%%%%%%%%%%%%%%%%%%%%%%%%%%%%%%%%%%%%%%%%%%%%%%%%%%%%%%%%%%%%%%%%%%%%%%%%%%%%%%%%%%%%%%%%%%%%%%%%%%%%%%%%%%%%%%%%%%%%%%%%%%%%%%%%%%%%%%%%%%%
%%%%%%%%%%%%%%%%%%%%%%%%%%%%%%%%%%%%%%%%%%%%%%%%%%%%%%%%%%%%%%%%%%%%%%%%%%%%%%%%%%%%%%%%%%%%%%%%%%%%%%%%%%%%%%%%%%%%%%%%%%%%%%%%%%%%%%%%%%%%%%%
%%%%%%%%%%%%%%%%%%%%%%%%%%%%%%%%%%%%%%%%%%%%% FIGURE 6 %%%%%%%%%%%%%%%%%%%%%%%%%%%%%%%%%%%%%%%%%%%%%%%%%%%%%%%%%%%%%%%%%%%%%%%%%%%%%%%%%%%%%%%%
%%%%%%%%%%%%%%%%%%%%%%%%%%%%%%%%%%%%%%%%%%%%%%%%%%%%%%%%%%%%%%%%%%%%%%%%%%%%%%%%%%%%%%%%%%%%%%%%%%%%%%%%%%%%%%%%%%%%%%%%%%%%%%%%%%%%%%%%%%%%%%%
%%%%%%%%%%%%%%%%%%%%%%%%%%%%%%%%%%%%%%%%%%%%%%%%%%%%%%%%%%%%%%%%%%%%%%%%%%%%%%%%%%%%%%%%%%%%%%%%%%%%%%%%%%%%%%%%%%%%%%%%%%%%%%%%%%%%%%%%%%%%%%%




%%%%%%%%%%%%%%%%%%%%%%%%%%%%%%%%%%%%%%%%%%%%%%%%%%%%%%%%%%%%%%%%%%%%%%%%%%%%%%%%%%%%%%%%%%%%%%%%%%%%%%%%%%%%%%%%%%%%%%%%%%%%%%%%%%%%%%%%%%%%%%%
%%%%%%%%%%%%%%%%%%%%%%%%%%%%%%%%%%%%%%%%%%%%%%%%%%%%%%%%%%%%%%%%%%%%%%%%%%%%%%%%%%%%%%%%%%%%%%%%%%%%%%%%%%%%%%%%%%%%%%%%%%%%%%%%%%%%%%%%%%%%%%%
%%%%%%%%%%%%%%%%%%%%%%%%%%%%%%%%%%%%%%%%%%%%% FIGURE 7 %%%%%%%%%%%%%%%%%%%%%%%%%%%%%%%%%%%%%%%%%%%%%%%%%%%%%%%%%%%%%%%%%%%%%%%%%%%%%%%%%%%%%%%%
%%%%%%%%%%%%%%%%%%%%%%%%%%%%%%%%%%%%%%%%%%%%%%%%%%%%%%%%%%%%%%%%%%%%%%%%%%%%%%%%%%%%%%%%%%%%%%%%%%%%%%%%%%%%%%%%%%%%%%%%%%%%%%%%%%%%%%%%%%%%%%%
%%%%%%%%%%%%%%%%%%%%%%%%%%%%%%%%%%%%%%%%%%%%%%%%%%%%%%%%%%%%%%%%%%%%%%%%%%%%%%%%%%%%%%%%%%%%%%%%%%%%%%%%%%%%%%%%%%%%%%%%%%%%%%%%%%%%%%%%%%%%%%%

\begin{figure}


    %%%%%%%%%%%%%%%%%%%%%%%%%%%%%%%%%%%%%%%%%%%%%%%%%%%%%%%%%%%%%%%%%%%%%%%%%%%%%%%%%%%%%%%%%%%%%%%%%%%%%%%%%%%%%%%%%%%%%%%%%%%%%%%%%%%%%%%%%%%
    %%%%%%%%%%%%%%%%%%%%%%%%%%%%%%%%%%%%%%%%%%%%% FIGURE 7 A %%%%%%%%%%%%%%%%%%%%%%%%%%%%%%%%%%%%%%%%%%%%%%%%%%%%%%%%%%%%%%%%%%%%%%%%%%%%%%%%%%
    %%%%%%%%%%%%%%%%%%%%%%%%%%%%%%%%%%%%%%%%%%%%%%%%%%%%%%%%%%%%%%%%%%%%%%%%%%%%%%%%%%%%%%%%%%%%%%%%%%%%%%%%%%%%%%%%%%%%%%%%%%%%%%%%%%%%%%%%%%%


	\begin{subfigure}[b]{0.6\linewidth}
	\centering
	\resizebox{\linewidth}{!}{

\begin{tikzpicture}
    \node[draw=none, fill=none] at (0, 10) {\Huge A};
    \node[draw=none,fill=none] (R) at (.75,3.5) {\Huge R};
    \node[draw=none,fill=none] (A) at (.75,0.5) {\Huge A};
    \draw [<- >, very thick] (A) to node[anchor=west] {\huge{$J_0^{(z_J)}$}} (R);
    %\draw[<->] (A) to {v} (RA);% {V};
    %\draw[->, bend right=22.5] (\from) to node[fill=white] {$T_{\from \to}$}
    \draw [black, very thick] (0,0) to [square left brace ] (0,4);
    \draw [black, very thick] (1.5,0) to [square right brace] (1.5,4);

    
    \draw [black, very thick] (10,0) to [square left brace ] (10,4);
    \draw [black, very thick] (11.5,0) to [square right brace] (11.5,4);
        \node[draw=none,fill=none] (X) at (10.75,3.5) {\Huge X};
    \node[draw=none,fill=none] (Xca) at (10.75,0.5) {{\huge{X}$\cdot\rm \large{Ca^{2+}}$}};
    \draw [<->, very thick, name=K] (X) to node[anchor=west] {\huge{$K_d$}} (Xca);
    
    \draw [black, very thick] (5,5) to [square left brace ] (5,9);
    \draw [black, very thick] (6.5,5) to [square right brace] (6.5,9);
    
    \node[draw=none,fill=none] (C) at (5.75,8.5) {\Huge C};
    \node[draw=none,fill=none] (O) at (5.75,5.5) {\huge{O}};
    \draw [<->, very thick, name=L] (C) to node[anchor=west] {\huge ${L}_0^{{{(\rcancel{z_L})}}}$} (O);
        \node [draw=none,fill=none] (four) at (2.3,0) {\huge 4};
    
	    \node [draw=none,fill=none] (four) at (12.3,0) {\huge 4};
    
    %D should be at (2.5+.75,7) $5.75+2.5=8.25 10.75-2.5 = 
    
    %guide to locations:
    	%J_v node is at (1.8,2)
        %K node is at (10.8,2)
        %L node is at (6.25,6.8)
        %D node is at (2.5,6)
        %C node is at (9,6)
        %E node is at 5.75,2
    
    
    \node[draw=none,fill=none,rotate=45] (D) at (2.5,6) {\Huge $ {D} $};
    
    \node[draw=none,fill=none,rotate=45, left of=D, xshift=.2cm] (<D) at (2.5,6) {\Huge $<$};
    
     \node[draw=none,fill=none,rotate=45, right of=D, xshift=-.2cm] (D>) at (2.5,6) {\Huge $>$};    
     
         \node[draw=none,fill=none,rotate=-45] (C) at (9,6) {\Huge $   {C} $};

         
    \node[draw=none,fill=none,rotate=45, above of=C, yshift=-.2cm] (<C) at (9,6) {\Huge $\wedge$};
    
     \node[draw=none,fill=none,rotate=45, below of=C, yshift=.2cm] (C>) at (9,6) {\Huge $\vee$};    

    
    \node[draw=none,fill=none] (E) at (5.75,2) {\Huge $ {E} $};
    
    
        \node[draw=none,fill=none, left of=E, xshift=.2cm] (<E) at (5.75,2) {\Huge $<$};
    
     \node[draw=none,fill=none, right of=E, xshift=-.2cm] (E>) at (5.75,2) {\Huge $>$};    
    
 %   \draw [-, line width=0.7mm, red, bend left=15,dashed] (6.25,6.8) to node[anchor=east,font=\bf] {\huge$\phi_1=\sqrt[4]{L_0}E$} (5.75,2);
    
%    \draw [-, line width=0.7mm, red, dashed] (6.25,7.25) .. controls +(up:6cm) and +(left:1cm) .. node[above,sloped,font=\bf] {\huge$\mathbf \phi_2=\sqrt[4]{L_0}D$} (2.5,6);
    
 %       \draw [-, line width=0.7mm, red, dashed] (6.25,7.25) .. controls +(up:6cm) and +(right:1cm) .. node[above,sloped,font=\bf] {\huge$\phi_3=\sqrt[4]{L_0}C$} (9,6);
\end{tikzpicture}
}

\end{subfigure}
\begin{subfigure}[b]{0.6\linewidth}
\centering
\begin{tikzpicture}
    \matrix(dict)[matrix of nodes,%below=of game,
        nodes={align=center,text width=1.5cm},
        row 1/.style={anchor=south},
        column 1/.style={nodes={text width=.9cm,align=right}}
    ]{
        $\theta_i$ & Relative error (\%)\\
        $L_0$ & $1.5\times 10^{22}$\\
        $J_0$ & $92.67$\\
        $z_J$ & $59.82$\\
        $K_d$ & $128.44$\\
        $C$ & $1.82\times 10^{6}$\\
        $D$ & $1.85\times 10^{7}$\\
        $E$ & $7.58\times 10^{6}$\\
    };
    \draw(dict-1-1.south west)--(dict-1-2.south east);
    \draw(dict-1-1.north east)--(dict-8-1.south east);
\end{tikzpicture}
\end{subfigure}

\vspace{1cm}

    %%%%%%%%%%%%%%%%%%%%%%%%%%%%%%%%%%%%%%%%%%%%%%%%%%%%%%%%%%%%%%%%%%%%%%%%%%%%%%%%%%%%%%%%%%%%%%%%%%%%%%%%%%%%%%%%%%%%%%%%%%%%%%%%%%%%%%%%%%%
    %%%%%%%%%%%%%%%%%%%%%%%%%%%%%%%%%%%%%%%%%%%%% FIGURE 7 B %%%%%%%%%%%%%%%%%%%%%%%%%%%%%%%%%%%%%%%%%%%%%%%%%%%%%%%%%%%%%%%%%%%%%%%%%%%%%%%%%%
    %%%%%%%%%%%%%%%%%%%%%%%%%%%%%%%%%%%%%%%%%%%%%%%%%%%%%%%%%%%%%%%%%%%%%%%%%%%%%%%%%%%%%%%%%%%%%%%%%%%%%%%%%%%%%%%%%%%%%%%%%%%%%%%%%%%%%%%%%%%

	\begin{subfigure}[b]{0.6\linewidth}
	\resizebox{\linewidth}{!}{

\begin{tikzpicture}
    \node[draw=none, fill=none] at (0, 10) {\Huge B};
    \node[draw=none,fill=none] (R) at (.75,3.5) {\Huge R};
    \node[draw=none,fill=none] (A) at (.75,0.5) {\Huge A};
    \draw [<- >, very thick] (A) to node[anchor=west] {\huge{${J}_0^{{(z_J)}}$}} (R);
    %\draw[<->] (A) to {v} (RA);% {V};
    %\draw[->, bend right=22.5] (\from) to node[fill=white] {$T_{\from \to}$}
    \draw [black, very thick] (0,0) to [square left brace ] (0,4);
    \draw [black, very thick] (1.5,0) to [square right brace] (1.5,4);
    
    \node[draw=none,fill=none,] (phione) at (.75,7.25) {\Huge $\phi_1 = \sqrt[4]{\Lo}D$};
    
    \node[draw=none,fill=none,] (phione) at (10.75,7.25) {\Huge $\phi_2 = \sqrt[4]{\Lo}{C}$};
    
    \node[draw=none,fill=none,] (phione) at (5.75,1) {\Huge $\phi_3 = \left.{E}\middle/\sqrt[4]{\Lo}\right.$};

    
    \draw [black, very thick] (10,0) to [square left brace ] (10,4);
    \draw [black, very thick] (11.5,0) to [square right brace] (11.5,4);
        \node[draw=none,fill=none] (X) at (10.75,3.5) {\Huge X};
    \node[draw=none,fill=none] (Xca) at (10.75,0.5) {{\huge{X}$\cdot\rm \large{Ca^{2+}}$}};
    \draw [<->, very thick, name=K] (X) to node[anchor=west] {\huge{${{K_d}}$}} (Xca);
    
    \draw [black, very thick] (5,5) to [square left brace ] (5,9);
    \draw [black, very thick] (6.5,5) to [square right brace] (6.5,9);
    
    \node[draw=none,fill=none] (C) at (5.75,8.5) {\Huge C};
    \node[draw=none,fill=none] (O) at (5.75,5.5) {\huge{O}};
    \draw [<->, very thick, name=L] (C) to node[anchor=west] {\huge $\rcancel[mediumcandyapplered]{{L}}_0^{{\rcancel[mediumcandyapplered]{(z_L)}}}$} (O);
        \node [draw=none,fill=none] (four) at (2.3,0) {\huge 4};
    
	    \node [draw=none,fill=none] (four) at (12.3,0) {\huge 4};
    
    %D should be at (2.5+.75,7) $5.75+2.5=8.25 10.75-2.5 = 
    
    %guide to locations:
    	%J_v node is at (1.8,2)
        %K node is at (10.8,2)
        %L node is at (6.25,6.8)
        %D node is at (2.5,6)
        %C node is at (9,6)
        %E node is at 5.75,2
    
    
    \node[draw=none,fill=none,rotate=45] (D) at (2.5,6) {\Huge $ {D} $};
    
    \node[draw=none,fill=none,rotate=45, left of=D, xshift=.2cm] (<D) at (2.5,6) {\Huge $<$};
    
     \node[draw=none,fill=none,rotate=45, right of=D, xshift=-.2cm] (D>) at (2.5,6) {\Huge $>$};    
     
         \node[draw=none,fill=none,rotate=-45] (C) at (9,6) {\Huge $   {C} $};

         
    \node[draw=none,fill=none,rotate=45, above of=C, yshift=-.2cm] (<C) at (9,6) {\Huge $\wedge$};
    
     \node[draw=none,fill=none,rotate=45, below of=C, yshift=.2cm] (C>) at (9,6) {\Huge $\vee$};    

    
    \node[draw=none,fill=none] (E) at (5.75,2) {\Huge $ {E} $};
    
    
        \node[draw=none,fill=none, left of=E, xshift=.2cm] (<E) at (5.75,2) {\Huge $<$};
    
     \node[draw=none,fill=none, right of=E, xshift=-.2cm] (E>) at (5.75,2) {\Huge $>$};   
              \draw[very thick,mediumcandyapplered,line width=1.25pt] (D.north west)--(D.south east);
        \draw[very thick,mediumcandyapplered,line width=1.25pt] (D.north east)--(D.south west); 
        
                      \draw[very thick,mediumcandyapplered,line width=1.25pt] (C.north west)--(C.south east);
        \draw[ very thick,mediumcandyapplered,line width=1.25pt] (C.north east)--(C.south west); 
        
                              \draw[very thick,mediumcandyapplered,line width=1.25pt] (E.north west)--(E.south east);
        \draw[very thick,mediumcandyapplered,line width=1.25pt] (E.north east)--(E.south west); 
    
    
 %   \draw [-, line width=0.7mm, red, bend left=15,dashed] (6.25,6.8) to node[anchor=east,font=\bf] {\huge$\phi_1=\sqrt[4]{L_0}E$} (5.75,2);
    
%    \draw [-, line width=0.7mm, red, dashed] (6.25,7.25) .. controls +(up:6cm) and +(left:1cm) .. node[above,sloped,font=\bf] {\huge$\mathbf \phi_2=\sqrt[4]{L_0}D$} (2.5,6);
    
 %       \draw [-, line width=0.7mm, red, dashed] (6.25,7.25) .. controls +(up:6cm) and +(right:1cm) .. node[above,sloped,font=\bf] {\huge$\phi_3=\sqrt[4]{L_0}C$} (9,6);
\end{tikzpicture}
}

\end{subfigure}
\begin{subfigure}[b]{0.6\linewidth}
\centering
\begin{tikzpicture}
    \matrix(dict)[matrix of nodes,%below=of game,
        nodes={align=center,text width=1.5cm},
        row 1/.style={anchor=south},
        column 1/.style={nodes={text width=.9cm,align=right}}
    ]{
        $\theta_i$ & Relative error (\%)\\
        $J_0$ & $82.35$\\
        $z_J$ & $36.15$\\
        $K_d$ & $51.03$\\
        $\sqrt[4]{L_0}C$ & $1.23\times 10^{9}$\\
        $\sqrt[4]{L_0}D$ & $96.86$\\
        $\frac{E}{\sqrt[4]{L_0}}$ & $1.31\times 10^{9}$\\
    };
    \draw(dict-1-1.south west)--(dict-1-2.south east);
    \draw(dict-1-1.north east)--(dict-7-1.south east);
\end{tikzpicture}
\end{subfigure}

\vspace{1cm}

    %%%%%%%%%%%%%%%%%%%%%%%%%%%%%%%%%%%%%%%%%%%%%%%%%%%%%%%%%%%%%%%%%%%%%%%%%%%%%%%%%%%%%%%%%%%%%%%%%%%%%%%%%%%%%%%%%%%%%%%%%%%%%%%%%%%%%%%%%%%
    %%%%%%%%%%%%%%%%%%%%%%%%%%%%%%%%%%%%%%%%%%%%% FIGURE 7 C %%%%%%%%%%%%%%%%%%%%%%%%%%%%%%%%%%%%%%%%%%%%%%%%%%%%%%%%%%%%%%%%%%%%%%%%%%%%%%%%%%
    %%%%%%%%%%%%%%%%%%%%%%%%%%%%%%%%%%%%%%%%%%%%%%%%%%%%%%%%%%%%%%%%%%%%%%%%%%%%%%%%%%%%%%%%%%%%%%%%%%%%%%%%%%%%%%%%%%%%%%%%%%%%%%%%%%%%%%%%%%%

	\begin{subfigure}[b]{0.6\linewidth}
	\resizebox{\linewidth}{!}{

\begin{tikzpicture}
    \node[draw=none, fill=none] at (0, 10) {\Huge C};
    \node[draw=none,fill=none] (R) at (.75,3.5) {\Huge R};
    \node[draw=none,fill=none] (A) at (.75,0.5) {\Huge A};
    \draw [<- >, very thick] (A) to node[anchor=west] {\huge{${J}_0^{{(z_J)}}$}} (R);
    %\draw[<->] (A) to {v} (RA);% {V};
    %\draw[->, bend right=22.5] (\from) to node[fill=white] {$T_{\from \to}$}
    \draw [black, very thick] (0,0) to [square left brace ] (0,4);
    \draw [black, very thick] (1.5,0) to [square right brace] (1.5,4);
    
    \node[draw=none,fill=none,] (phione) at (.75,7.25) {\Huge $\phi_1 = \sqrt[4]{\Lo}D$};
    
    \node[draw=none,fill=none,] (phione) at (10.75,7.25) {\Huge $\rcancel{\phi_2 = \sqrt[4]{\Lo}{C}}$};
    
    \node[draw=none,fill=none,] (phione) at (5.75,1) {\Huge $\rcancel{\phi_3 = \left.{E}\middle/\sqrt[4]{\Lo}\right.}$};
    
     \node[draw=none,fill=none,] (phione) at (6.9,3.75) {\Huge ${\phi_4 = {CE}}$};

    
    \draw [black, very thick] (10,0) to [square left brace ] (10,4);
    \draw [black, very thick] (11.5,0) to [square right brace] (11.5,4);
        \node[draw=none,fill=none] (X) at (10.75,3.5) {\Huge X};
    \node[draw=none,fill=none] (Xca) at (10.75,0.5) {{\huge{X}$\cdot\rm \large{Ca^{2+}}$}};
    \draw [<->, very thick, name=K] (X) to node[anchor=west] {\huge{${{K_d}}$}} (Xca);
    
    \draw [black, very thick] (5,5) to [square left brace ] (5,9);
    \draw [black, very thick] (6.5,5) to [square right brace] (6.5,9);
    
    \node[draw=none,fill=none] (C) at (5.75,8.5) {\Huge C};
    \node[draw=none,fill=none] (O) at (5.75,5.5) {\huge{O}};
    \draw [<->, very thick, name=L] (C) to node[anchor=west] {\huge $\rcancel[mediumcandyapplered]{{L}}_0^{{\rcancel[mediumcandyapplered]{(z_L)}}}$} (O);
        \node [draw=none,fill=none] (four) at (2.3,0) {\huge 4};
    
	    \node [draw=none,fill=none] (four) at (12.3,0) {\huge 4};
    
    %D should be at (2.5+.75,7) $5.75+2.5=8.25 10.75-2.5 = 
    
    %guide to locations:
    	%J_v node is at (1.8,2)
        %K node is at (10.8,2)
        %L node is at (6.25,6.8)
        %D node is at (2.5,6)
        %C node is at (9,6)
        %E node is at 5.75,2
    
    
    \node[draw=none,fill=none,rotate=45] (D) at (2.5,6) {\Huge $ {D} $};
    
    \node[draw=none,fill=none,rotate=45, left of=D, xshift=.2cm] (<D) at (2.5,6) {\Huge $<$};
    
     \node[draw=none,fill=none,rotate=45, right of=D, xshift=-.2cm] (D>) at (2.5,6) {\Huge $>$};    
     
         \node[draw=none,fill=none,rotate=-45] (C) at (9,6) {\Huge $   {C} $};

         
    \node[draw=none,fill=none,rotate=45, above of=C, yshift=-.2cm] (<C) at (9,6) {\Huge $\wedge$};
    
     \node[draw=none,fill=none,rotate=45, below of=C, yshift=.2cm] (C>) at (9,6) {\Huge $\vee$};    

    
    \node[draw=none,fill=none] (E) at (5.75,2) {\Huge $ {E} $};
    
    
        \node[draw=none,fill=none, left of=E, xshift=.2cm] (<E) at (5.75,2) {\Huge $<$};
    
     \node[draw=none,fill=none, right of=E, xshift=-.2cm] (E>) at (5.75,2) {\Huge $>$};   
              \draw[very thick,mediumcandyapplered,line width=1.25pt] (D.north west)--(D.south east);
        \draw[very thick,mediumcandyapplered,line width=1.25pt] (D.north east)--(D.south west); 
        
                      \draw[very thick,mediumcandyapplered,line width=1.25pt] (C.north west)--(C.south east);
        \draw[ very thick,mediumcandyapplered,line width=1.25pt] (C.north east)--(C.south west); 
        
                              \draw[very thick,mediumcandyapplered,line width=1.25pt] (E.north west)--(E.south east);
        \draw[very thick,mediumcandyapplered,line width=1.25pt] (E.north east)--(E.south west); 
    
    
 %   \draw [-, line width=0.7mm, red, bend left=15,dashed] (6.25,6.8) to node[anchor=east,font=\bf] {\huge$\phi_1=\sqrt[4]{L_0}E$} (5.75,2);
    
%    \draw [-, line width=0.7mm, red, dashed] (6.25,7.25) .. controls +(up:6cm) and +(left:1cm) .. node[above,sloped,font=\bf] {\huge$\mathbf \phi_2=\sqrt[4]{L_0}D$} (2.5,6);
    
 %       \draw [-, line width=0.7mm, red, dashed] (6.25,7.25) .. controls +(up:6cm) and +(right:1cm) .. node[above,sloped,font=\bf] {\huge$\phi_3=\sqrt[4]{L_0}C$} (9,6);
\end{tikzpicture}
}

\end{subfigure}
\begin{subfigure}[b]{0.6\linewidth}
\centering
\begin{tikzpicture}
    \matrix(dict)[matrix of nodes,%below=of game,
        nodes={align=center,text width=1.5cm},
        row 1/.style={anchor=south},
        column 1/.style={nodes={text width=.9cm,align=right}}
    ]{
        $\theta_i$ & Relative error (\%)\\
        $J_0$ & $84.52$\\
        $z_J$ & $17.99$\\
        $K_d$ & $47.89$\\
        $CE$ & $44.78$\\
        $\sqrt[4]{L_0}D$ & $68.67$\\
    };
    \draw(dict-1-1.south west)--(dict-1-2.south east);
    \draw(dict-1-1.north east)--(dict-6-1.south east);
\end{tikzpicture}
\end{subfigure}

\captionsetup{width=1.2\linewidth}
\caption{Model reduction results in identifiable parameters.  Reduced models for the $\po$ assay are presented in the left column, and the error in their parameters (95\% confidence interval) are presented at right.  (A),(B),(C) correspond to model-error pairs after one, two, and three reduction steps, respectively. The five parameter model produced by three model reductions (C, left) has identifiable parameters (within one order of magnitude for all parameters, right).}



\end{figure}

\begin{figure}


    %%%%%%%%%%%%%%%%%%%%%%%%%%%%%%%%%%%%%%%%%%%%%%%%%%%%%%%%%%%%%%%%%%%%%%%%%%%%%%%%%%%%%%%%%%%%%%%%%%%%%%%%%%%%%%%%%%%%%%%%%%%%%%%%%%%%%%%%%%%
    %%%%%%%%%%%%%%%%%%%%%%%%%%%%%%%%%%%%%%%%%%%%% FIGURE 6 A %%%%%%%%%%%%%%%%%%%%%%%%%%%%%%%%%%%%%%%%%%%%%%%%%%%%%%%%%%%%%%%%%%%%%%%%%%%%%%%%%%
    %%%%%%%%%%%%%%%%%%%%%%%%%%%%%%%%%%%%%%%%%%%%%%%%%%%%%%%%%%%%%%%%%%%%%%%%%%%%%%%%%%%%%%%%%%%%%%%%%%%%%%%%%%%%%%%%%%%%%%%%%%%%%%%%%%%%%%%%%%%


%\begin{center}
%	\begin{subfigure}[b]{0.48\linewidth}
%	\centering
%	\resizebox{\linewidth}{!}{
%
%\begin{tikzpicture}
%\node[draw=none, fill=none] at (-0.9,6.5) {\Large A};
%\begin{axis}
%       [
%         axis line style = { draw = none },
%         %ymode=log,
%        % xtick           = {-100,0,160}         ,
%         %ytick           = {0,.5,1}         ,
%         tick pos        = left           ,
%         xlabel={reduction step},
%         ylabel = {RMS error},
%       ]
%
%\pgfplotsset{cycle list set=0}
%\addplot table [x index=0, y index=1, col sep=comma] {../CSV/figure_6a_panel1.csv};
%
%\end{axis}
%
%\end{tikzpicture}
%}
%
%\end{subfigure}
%\end{center}
%\vspace{1cm}
    %%%%%%%%%%%%%%%%%%%%%%%%%%%%%%%%%%%%%%%%%%%%%%%%%%%%%%%%%%%%%%%%%%%%%%%%%%%%%%%%%%%%%%%%%%%%%%%%%%%%%%%%%%%%%%%%%%%%%%%%%%%%%%%%%%%%%%%%%%%
    %%%%%%%%%%%%%%%%%%%%%%%%%%%%%%%%%%%%%%%%%%%%% FIGURE 6 B left %%%%%%%%%%%%%%%%%%%%%%%%%%%%%%%%%%%%%%%%%%%%%%%%%%%%%%%%%%%%%%%%%%%%%%%%%%%%%
    %%%%%%%%%%%%%%%%%%%%%%%%%%%%%%%%%%%%%%%%%%%%%%%%%%%%%%%%%%%%%%%%%%%%%%%%%%%%%%%%%%%%%%%%%%%%%%%%%%%%%%%%%%%%%%%%%%%%%%%%%%%%%%%%%%%%%%%%%%%
    \hspace{-1.5cm}
	\begin{subfigure}[b]{0.6\linewidth}
	\centering
	\resizebox{\linewidth}{!}{

\begin{tikzpicture}
    \node[draw=none, fill=none] at (0, 10) {\Huge A};
    \node[draw=none,fill=none] (R) at (.75,3.5) {\Huge R};
    \node[draw=none,fill=none] (A) at (.75,0.5) {\Huge A};
    \draw [<- >, very thick] (A) to node[anchor=west] {\huge{$J_0^{(z_J)}$}} (R);
    %\draw[<->] (A) to {v} (RA);% {V};
    %\draw[->, bend right=22.5] (\from) to node[fill=white] {$T_{\from \to}$}
    \draw [black, very thick] (0,0) to [square left brace ] (0,4);
    \draw [black, very thick] (1.5,0) to [square right brace] (1.5,4);

    
    \draw [black, very thick] (10,0) to [square left brace ] (10,4);
    \draw [black, very thick] (11.5,0) to [square right brace] (11.5,4);
        \node[draw=none,fill=none] (X) at (10.75,3.5) {\Huge X};
    \node[draw=none,fill=none] (Xca) at (10.75,0.5) {{\huge{X}$\cdot\rm \large{Ca^{2+}}$}};
    \draw [<->, very thick, name=K] (X) to node[anchor=west] {\huge{$K_d$}} (Xca);
    
    \draw [black, very thick] (5,5) to [square left brace ] (5,9);
    \draw [black, very thick] (6.5,5) to [square right brace] (6.5,9);
    
    \node[draw=none,fill=none] (C) at (5.75,8.5) {\Huge C};
    \node[draw=none,fill=none] (O) at (5.75,5.5) {\huge{O}};
    \draw [<->, very thick, name=L] (C) to node[anchor=west] {\huge ${L}_0^{{{(\rcancel{z_L})}}}$} (O);
        \node [draw=none,fill=none] (four) at (2.3,0) {\huge 4};
    
	    \node [draw=none,fill=none] (four) at (12.3,0) {\huge 4};
    
    %D should be at (2.5+.75,7) $5.75+2.5=8.25 10.75-2.5 = 
    
    %guide to locations:
    	%J_v node is at (1.8,2)
        %K node is at (10.8,2)
        %L node is at (6.25,6.8)
        %D node is at (2.5,6)
        %C node is at (9,6)
        %E node is at 5.75,2
    
    
    \node[draw=none,fill=none,rotate=45] (D) at (2.5,6) {\Huge $ {D} $};
    
    \node[draw=none,fill=none,rotate=45, left of=D, xshift=.2cm] (<D) at (2.5,6) {\Huge $<$};
    
     \node[draw=none,fill=none,rotate=45, right of=D, xshift=-.2cm] (D>) at (2.5,6) {\Huge $>$};    
     
         \node[draw=none,fill=none,rotate=-45] (C) at (9,6) {\Huge $   {C} $};

         
    \node[draw=none,fill=none,rotate=45, above of=C, yshift=-.2cm] (<C) at (9,6) {\Huge $\wedge$};
    
     \node[draw=none,fill=none,rotate=45, below of=C, yshift=.2cm] (C>) at (9,6) {\Huge $\vee$};    

    
    \node[draw=none,fill=none] (E) at (5.75,2) {\Huge $ {E} $};
    
    
        \node[draw=none,fill=none, left of=E, xshift=.2cm] (<E) at (5.75,2) {\Huge $<$};
    
     \node[draw=none,fill=none, right of=E, xshift=-.2cm] (E>) at (5.75,2) {\Huge $>$};    
    
 %   \draw [-, line width=0.7mm, red, bend left=15,dashed] (6.25,6.8) to node[anchor=east,font=\bf] {\huge$\phi_1=\sqrt[4]{L_0}E$} (5.75,2);
    
%    \draw [-, line width=0.7mm, red, dashed] (6.25,7.25) .. controls +(up:6cm) and +(left:1cm) .. node[above,sloped,font=\bf] {\huge$\mathbf \phi_2=\sqrt[4]{L_0}D$} (2.5,6);
    
 %       \draw [-, line width=0.7mm, red, dashed] (6.25,7.25) .. controls +(up:6cm) and +(right:1cm) .. node[above,sloped,font=\bf] {\huge$\phi_3=\sqrt[4]{L_0}C$} (9,6);
\end{tikzpicture}
}

\end{subfigure}
    %%%%%%%%%%%%%%%%%%%%%%%%%%%%%%%%%%%%%%%%%%%%%%%%%%%%%%%%%%%%%%%%%%%%%%%%%%%%%%%%%%%%%%%%%%%%%%%%%%%%%%%%%%%%%%%%%%%%%%%%%%%%%%%%%%%%%%%%%%%
    %%%%%%%%%%%%%%%%%%%%%%%%%%%%%%%%%%%%%%%%%%%%% FIGURE 6 B right %%%%%%%%%%%%%%%%%%%%%%%%%%%%%%%%%%%%%%%%%%%%%%%%%%%%%%%%%%%%%%%%%%%%%%%%%%%%
    %%%%%%%%%%%%%%%%%%%%%%%%%%%%%%%%%%%%%%%%%%%%%%%%%%%%%%%%%%%%%%%%%%%%%%%%%%%%%%%%%%%%%%%%%%%%%%%%%%%%%%%%%%%%%%%%%%%%%%%%%%%%%%%%%%%%%%%%%%%
\begin{subfigure}[b]{0.6\linewidth}

	\resizebox{\linewidth}{!}{
\begin{tikzpicture}
\begin{axis}
       [
         axis line style = { draw = none },
         %ymode=log,
        % xtick           = {-100,0,160}         ,
         %ytick           = {0,.5,1}         ,
         tick pos        = left           ,
         xlabel={V (mv)},
         ylabel = {$P_o$},
       ]
       
\pgfplotsset{cycle list set=0}
\foreach \i/\j in \CaSTR
{
\addplot +[only marks] table [x index=0, y index=1, col sep=comma] {../CSV/figure_1c_panel1_ca\i.csv};\label{\j}
}
\pgfplotsset{cycle list set=0}
\foreach \i/\j in \CaSTR
{
\addplot +[no markers] table [x index=0, y index=1, col sep=comma, mark=circle]{../CSV/figure_6b_panel2_ca\i.csv};
}


%\addplot table [x index=0, y index=0, col sep=comma] {figure_3b_panel1_paramzJ.csv};
%\addplot table [x=trila, y=st]
    %     \node [left] at (axis cs:  6,6) {$32hhhh$};
\end{axis}

\end{tikzpicture}
}

\end{subfigure}

\vspace{1cm}

    \hspace{-1.5cm}
\begin{subfigure}[b]{0.48\linewidth}
\centering
\begin{tikzpicture}
    \node[draw=none, fill=none] at (-7.5, 1.2) {\Large B};
    \matrix(dict)[matrix of nodes,%below=of game,
        nodes={align=center,text width=1.25cm},
        row 1/.style={anchor=south},
        row 2/.style={color=black},
        column 1/.style={nodes={text width=4.0cm,align=right}}
    ]{
        Parameter ($\theta_i)$ & $L_0$ & $J_0$ & $z_J$ & $K_d$ & $C$ & $D$ & $E$\\
         %$P_o$ Relative error (\%)  & $5\times10^{11}$ & $4.0\times10^4$ & $7.4\times10^3$ & $71.80$ & $304.31$ &  $2.8\times10^3$ & $1.8\times10^6$ & $1.7\times10^4$\\
        Relative error (\%)  & $68$ & $43$ & $5.6$ & $11$ &  $16$ & $52$ & $23$\\
    };
    \draw(dict-1-1.south west)--(dict-1-8.south east);
    \draw(dict-1-1.north east)--(dict-2-1.south east);
    % Data is in {../CSV/figure_3a_panel1.csv}
\end{tikzpicture}
\end{subfigure}


    \vspace{1cm}
    %%%%%%%%%%%%%%%%%%%%%%%%%%%%%%%%%%%%%%%%%%%%%%%%%%%%%%%%%%%%%%%%%%%%%%%%%%%%%%%%%%%%%%%%%%%%%%%%%%%%%%%%%%%%%%%%%%%%%%%%%%%%%%%%%%%%%%%%%%%
    %%%%%%%%%%%%%%%%%%%%%%%%%%%%%%%%%%%%%%%%%%%%% FIGURE 6 C left %%%%%%%%%%%%%%%%%%%%%%%%%%%%%%%%%%%%%%%%%%%%%%%%%%%%%%%%%%%%%%%%%%%%%%%%%%%%%
    %%%%%%%%%%%%%%%%%%%%%%%%%%%%%%%%%%%%%%%%%%%%%%%%%%%%%%%%%%%%%%%%%%%%%%%%%%%%%%%%%%%%%%%%%%%%%%%%%%%%%%%%%%%%%%%%%%%%%%%%%%%%%%%%%%%%%%%%%%%
    
\captionsetup{width=1.2\linewidth}
\caption{Single reduction identifiability for $\lpo$.  A model resulting from one reduction (A, left) fits synthetic data (A, right, data legend as in Fig 1) very well (solid lines).  This model has identifiable parameters (B).}

\end{figure}



%%%%%%%%%%%%%%%%%%%%%%%%%%%%%%%%%%%%%%%%%%%%%%%%%%%%%%%%%%%%%%%%%%%%%%%%%%%%%%%%%%%%%%%%%%%%%%%%%%%%%%%%%%%%%%%%%%%%%%%%%%%%%%%%%%%%%%%%%%%%%%%
%%%%%%%%%%%%%%%%%%%%%%%%%%%%%%%%%%%%%%%%%%%%%%%%%%%%%%%%%%%%%%%%%%%%%%%%%%%%%%%%%%%%%%%%%%%%%%%%%%%%%%%%%%%%%%%%%%%%%%%%%%%%%%%%%%%%%%%%%%%%%%%
%%%%%%%%%%%%%%%%%%%%%%%%%%%%%%%%%%%%%%%%%%%%% FIGURE 8 %%%%%%%%%%%%%%%%%%%%%%%%%%%%%%%%%%%%%%%%%%%%%%%%%%%%%%%%%%%%%%%%%%%%%%%%%%%%%%%%%%%%%%%%
%%%%%%%%%%%%%%%%%%%%%%%%%%%%%%%%%%%%%%%%%%%%%%%%%%%%%%%%%%%%%%%%%%%%%%%%%%%%%%%%%%%%%%%%%%%%%%%%%%%%%%%%%%%%%%%%%%%%%%%%%%%%%%%%%%%%%%%%%%%%%%%
%%%%%%%%%%%%%%%%%%%%%%%%%%%%%%%%%%%%%%%%%%%%%%%%%%%%%%%%%%%%%%%%%%%%%%%%%%%%%%%%%%%%%%%%%%%%%%%%%%%%%%%%%%%%%%%%%%%%%%%%%%%%%%%%%%%%%%%%%%%%%%%

\begin{figure}


%%%%%%%%%%%%%%%%%%%%%%%%%%%%%%%%%%%%%%%%%%%%%%%%%%%%%%%%%%%%%%%%%%%%%%%%%%%%%%%%%%%%%%%%%%%%%%%%%%%%%%%%%%%%%%%%%%%%%%%%%%%%%%%%%%%%%%%%%%%
%%%%%%%%%%%%%%%%%%%%%%%%%%%%%%%%%%%%%%%%%%%%% FIGURE 8 A %%%%%%%%%%%%%%%%%%%%%%%%%%%%%%%%%%%%%%%%%%%%%%%%%%%%%%%%%%%%%%%%%%%%%%%%%%%%%%%%%%
%%%%%%%%%%%%%%%%%%%%%%%%%%%%%%%%%%%%%%%%%%%%%%%%%%%%%%%%%%%%%%%%%%%%%%%%%%%%%%%%%%%%%%%%%%%%%%%%%%%%%%%%%%%%%%%%%%%%%%%%%%%%%%%%%%%%%%%%%%%

\hspace{-.9cm}
    \begin{subfigure}[b]{0.6\linewidth}
    \centering
    \resizebox{\linewidth}{!}{

\begin{tikzpicture}
\node[draw=none, fill=none] at (0.3,6.6) {\LARGE A};
\begin{axis}
       [
         axis line style = { draw = none },
         %ymode=log,
         xtick           = {0, 1}         ,
         %ytick           = {-10, -5, 0}  ,
         ymin = -9,
         ymax = -3 ,
         tick pos        = left           ,
         xlabel={},
         xticklabels    = {{Base}, {Test}},
         ylabel = {$\log_{10}(L_0)$},
       ]
       


\def\cursf{a}

\addplot +[only marks, color=black, mark=star] table [x index=0, y index=1, col sep=comma] {../CSV/figure_8\cursf_panel3.csv};
\addplot +[only marks, color=red, mark=star] table [x index=0, y index=1, col sep=comma] {../CSV/figure_8\cursf_panel4.csv};
\addplot +[only marks, color=blue, mark=star] table [x index=0, y index=1, col sep=comma] {../CSV/figure_8\cursf_panel5.csv};
\addplot[no markers, very thick, color=black, dashed ] table [x index=0, y index=1, col sep=comma] {../CSV/figure_8\cursf_panel1.csv};
\addplot[no markers, very thick, color=black] table [x index=0, y index=1, col sep=comma] {../CSV/figure_8\cursf_panel2.csv};

\end{axis}

\end{tikzpicture}
}

\end{subfigure}
%%%%%%%%%%%%%%%%%%%%%%%%%%%%%%%%%%%%%%%%%%%%%%%%%%%%%%%%%%%%%%%%%%%%%%%%%%%%%%%%%%%%%%%%%%%%%%%%%%%%%%%%%%%%%%%%%%%%%%%%%%%%%%%%%%%%%%%%%%%
    %%%%%%%%%%%%%%%%%%%%%%%%%%%%%%%%%%%%%%%%%%%%% FIGURE 8 B %%%%%%%%%%%%%%%%%%%%%%%%%%%%%%%%%%%%%%%%%%%%%%%%%%%%%%%%%%%%%%%%%%%%%%%%%%%%%%%%%%
    %%%%%%%%%%%%%%%%%%%%%%%%%%%%%%%%%%%%%%%%%%%%%%%%%%%%%%%%%%%%%%%%%%%%%%%%%%%%%%%%%%%%%%%%%%%%%%%%%%%%%%%%%%%%%%%%%%%%%%%%%%%%%%%%%%%%%%%%%%%
    \begin{subfigure}[b]{0.6\linewidth}
    \centering
    \resizebox{\linewidth}{!}{

\begin{tikzpicture}
\node[draw=none, fill=none] at (0.3,6.6) {\LARGE B};
\begin{axis}
       [
         axis line style = { draw = none },
         %ymode=log,
         xtick           = {0, 1}         ,
         %ytick           = {-50, -40, -30, -20, -10, 0, 10}  ,
         ymin = 0.5,
         ymax = 3.5 ,
         tick pos        = left           ,
         xlabel={},
         xticklabels     = {{Base}, {Test}},
         ylabel = {$\log_{10}(D)$},
       ]
       

\def\cursf{f}

\addplot +[only marks, color=black, mark=star] table [x index=0, y index=1, col sep=comma] {../CSV/figure_8\cursf_panel3.csv};
\addplot +[only marks, color=red, mark=star] table [x index=0, y index=1, col sep=comma] {../CSV/figure_8\cursf_panel4.csv};
\addplot +[only marks, color=blue, mark=star] table [x index=0, y index=1, col sep=comma] {../CSV/figure_8\cursf_panel5.csv};
\addplot[no markers, very thick, color=black, dashed ] table [x index=0, y index=1, col sep=comma] {../CSV/figure_8\cursf_panel1.csv};
\addplot[no markers, very thick, color=black] table [x index=0, y index=1, col sep=comma] {../CSV/figure_8\cursf_panel2.csv};

\end{axis}

\end{tikzpicture}
}



\end{subfigure}

\vspace{1cm}
\begin{center}
%%%%%%%%%%%%%%%%%%%%%%%%%%%%%%%%%%%%%%%%%%%%%%%%%%%%%%%%%%%%%%%%%%%%%%%%%%%%%%%%%%%%%%%%%%%%%%%%%%%%%%%%%%%%%%%%%%%%%%%%%%%%%%%%%%%%%%%%%%%
    %%%%%%%%%%%%%%%%%%%%%%%%%%%%%%%%%%%%%%%%%%%%% FIGURE 8 C %%%%%%%%%%%%%%%%%%%%%%%%%%%%%%%%%%%%%%%%%%%%%%%%%%%%%%%%%%%%%%%%%%%%%%%%%%%%%%%%%%
    %%%%%%%%%%%%%%%%%%%%%%%%%%%%%%%%%%%%%%%%%%%%%%%%%%%%%%%%%%%%%%%%%%%%%%%%%%%%%%%%%%%%%%%%%%%%%%%%%%%%%%%%%%%%%%%%%%%%%%%%%%%%%%%%%%%%%%%%%%%
    \begin{subfigure}[b]{0.6\linewidth}
    \centering
    \resizebox{\linewidth}{!}{

\begin{tikzpicture}
\node[draw=none, fill=none] at (0.3,6.6) {\LARGE C};
\begin{axis}
       [
         axis line style = { draw = none },
         %ymode=log,
         xtick           = {0, 1}         ,
         %tick           = {-10, 0, 10}         ,
         ymin = -2.5,
         ymax = -0.5 ,
         tick pos        = left           ,
         xlabel={},
         xticklabels    = {{Base}, {Test}},
         ylabel = {$\log_{10}(J_0)$},
       ]
       

\def\cursf{b}

\addplot +[only marks, color=black, mark=star] table [x index=0, y index=1, col sep=comma] {../CSV/figure_8\cursf_panel3.csv};
\addplot +[only marks, color=red, mark=star] table [x index=0, y index=1, col sep=comma] {../CSV/figure_8\cursf_panel4.csv};
\addplot +[only marks, color=blue, mark=star] table [x index=0, y index=1, col sep=comma] {../CSV/figure_8\cursf_panel5.csv};
\addplot[no markers, very thick, color=black, dashed ] table [x index=0, y index=1, col sep=comma] {../CSV/figure_8\cursf_panel1.csv};
\addplot[no markers, very thick, color=black] table [x index=0, y index=1, col sep=comma] {../CSV/figure_8\cursf_panel2.csv};

\end{axis}

\end{tikzpicture}
}

\end{subfigure}

\end{center}

\hspace{.2cm}
\captionsetup{width=1.2\linewidth}
\caption{Inferred parameters from the reduced $\lpo$ model.  Values inferred by fitting the once-reduced $\lpo$ model to 100 noisy synthetic measurements generated from base (black) and test (red) parameter sets.  (A),(B),(C) correspond to inferred values of $\jo, D, \jo,$ respectively.  Solid lines connect the means of the inferred base and test values; dotted lines connect the true generating base and test values.}



\end{figure}

\pagebreak

%%%%%%%%%%%%%%%%%%%%%%%%%%%%%%%%%%%%%%%%%%%%%%%%%%%%%%%%%%%%%%%%%%%%%%%%%%%%%%%%%%%%%%%%%%%%%%%%%%%%%%%%%%%%%%%%%%%%%%%%%%%%%%%%%%%%%%%%%%%%%%%
%%%%%%%%%%%%%%%%%%%%%%%%%%%%%%%%%%%%%%%%%%%%%%%%%%%%%%%%%%%%%%%%%%%%%%%%%%%%%%%%%%%%%%%%%%%%%%%%%%%%%%%%%%%%%%%%%%%%%%%%%%%%%%%%%%%%%%%%%%%%%%%
%%%%%%%%%%%%%%%%%%%%%%%%%%%%%%%%%%%%%%%%%%%%% FIGURE A1 %%%%%%%%%%%%%%%%%%%%%%%%%%%%%%%%%%%%%%%%%%%%%%%%%%%%%%%%%%%%%%%%%%%%%%%%%%%%%%%%%%%%%%%%
%%%%%%%%%%%%%%%%%%%%%%%%%%%%%%%%%%%%%%%%%%%%%%%%%%%%%%%%%%%%%%%%%%%%%%%%%%%%%%%%%%%%%%%%%%%%%%%%%%%%%%%%%%%%%%%%%%%%%%%%%%%%%%%%%%%%%%%%%%%%%%%
%%%%%%%%%%%%%%%%%%%%%%%%%%%%%%%%%%%%%%%%%%%%%%%%%%%%%%%%%%%%%%%%%%%%%%%%%%%%%%%%%%%%%%%%%%%%%%%%%%%%%%%%%%%%%%%%%%%%%%%%%%%%%%%%%%%%%%%%%%%%%%%

\begin{figure}

\vspace{1cm}

        \begin{subfigure}[b]{0.8\linewidth}
    \centering
    \resizebox{\linewidth}{!}{
\begin{tikzpicture}
\begin{axis}
       [
         axis line style = { draw = none },
         %xtick           = {-10,6}         ,
         %ytick           = {-.544727175441672, -13.027053197600004, -10.151948911834628, .6931471805599453,
    %     -.8675005677047232}         ,
         tick pos        = left           ,
                  xmax = 6.14,
       %  yticklabels={$z_J$, $L_0$, $K_d$, $E$, $z_L$},
         ylabel={$\log(\theta)$},
         xlabel = {time},
       ]


\pgfplotsset{cycle list set=0}
\foreach \i in {L0, J0, zJ, Kd, C, D, E}
{
\addplot table [x index=0, y index=1, col sep=comma] {../CSV/figure_5a_panel1_param_\i.csv};
};
\foreach \i in {zL}
{
\addplot table [x index=0, y index=1, col sep=comma] {../CSV/figure_5a_panel1_param_\i.csv};
};

\legend{$L_0$, $J_0$, $z_J$, $K_d$, $C$, $D$, $E$, $z_L$}

%\node[pos=0.0, pin=left:``first point'']{} ;




%\addplot table [x=trila, y=st]
        % \node [left] at (axis cs:  6,6) {$32hhhh$};
\end{axis}

\end{tikzpicture}
}
\end{subfigure}


\captionsetup{labelformat=empty,width=1.1\linewidth}
\caption{Figure 8 - supplement figure 1: Parameter compensations explain discrepancy between true and inferred values.  This figure shows parameter values during the first parameter reduction, which results in the model with one reduced parameter ($\zl$ eliminated).  Note that as $\zl$ goes to zero, $\lo$ and $\jo$ must decrease, and $D$ must increase, to compensate.  We therefore expect that when the once-reduced model is fit to data generated from the full model, $\lo, \jo$ will be smaller than the true parameters, and $D$ will be larger than the true parameters.  This is what we observe in Figure 8. }


\end{figure}

\addtocounter{figure}{-1}



%%%%%%%%%%%%%%%%%%%%%%%%%%%%%%%%%%%%%%%%%%%%%%%%%%%%%%%%%%%%%%%%%%%%%%%%%%%%%%%%%%%%%%%%%%%%%%%%%%%%%%%%%%%%%%%%%%%%%%%%%%%%%%%%%%%%%%%%%%%%%%%
%%%%%%%%%%%%%%%%%%%%%%%%%%%%%%%%%%%%%%%%%%%%%%%%%%%%%%%%%%%%%%%%%%%%%%%%%%%%%%%%%%%%%%%%%%%%%%%%%%%%%%%%%%%%%%%%%%%%%%%%%%%%%%%%%%%%%%%%%%%%%%%
%%%%%%%%%%%%%%%%%%%%%%%%%%%%%%%%%%%%%%%%%%%%% FIGURE A3 %%%%%%%%%%%%%%%%%%%%%%%%%%%%%%%%%%%%%%%%%%%%%%%%%%%%%%%%%%%%%%%%%%%%%%%%%%%%%%%%%%%%%%%%
%%%%%%%%%%%%%%%%%%%%%%%%%%%%%%%%%%%%%%%%%%%%%%%%%%%%%%%%%%%%%%%%%%%%%%%%%%%%%%%%%%%%%%%%%%%%%%%%%%%%%%%%%%%%%%%%%%%%%%%%%%%%%%%%%%%%%%%%%%%%%%%
%%%%%%%%%%%%%%%%%%%%%%%%%%%%%%%%%%%%%%%%%%%%%%%%%%%%%%%%%%%%%%%%%%%%%%%%%%%%%%%%%%%%%%%%%%%%%%%%%%%%%%%%%%%%%%%%%%%%%%%%%%%%%%%%%%%%%%%%%%%%%%%

\begin{figure}


    %%%%%%%%%%%%%%%%%%%%%%%%%%%%%%%%%%%%%%%%%%%%%%%%%%%%%%%%%%%%%%%%%%%%%%%%%%%%%%%%%%%%%%%%%%%%%%%%%%%%%%%%%%%%%%%%%%%%%%%%%%%%%%%%%%%%%%%%%%%
    %%%%%%%%%%%%%%%%%%%%%%%%%%%%%%%%%%%%%%%%%%%%% FIGURE A3 A %%%%%%%%%%%%%%%%%%%%%%%%%%%%%%%%%%%%%%%%%%%%%%%%%%%%%%%%%%%%%%%%%%%%%%%%%%%%%%%%%%
    %%%%%%%%%%%%%%%%%%%%%%%%%%%%%%%%%%%%%%%%%%%%%%%%%%%%%%%%%%%%%%%%%%%%%%%%%%%%%%%%%%%%%%%%%%%%%%%%%%%%%%%%%%%%%%%%%%%%%%%%%%%%%%%%%%%%%%%%%%%


    \begin{subfigure}[b]{0.50\linewidth}
    \centering
    \resizebox{\linewidth}{!}{

\begin{tikzpicture}

\begin{axis}
       [
         axis line style = { draw = none },
         %ymode=log,
         xtick           = {0, 1}         ,
         ytick           = {-10, 0, 10}         ,
         ymin = -14,
         ymax = 14 ,
         tick pos        = left           ,
         xlabel={},
         xticklabels    = {{Base}, {Test}},
         ylabel = {$\log_{10}(C)$},
       ]
       


\def\cursf{a}

\addplot +[only marks, color=black, mark=star] table [x index=0, y index=1, col sep=comma] {../CSV/figure_A3\cursf_panel3.csv};
\addplot +[only marks, color=red, mark=star] table [x index=0, y index=1, col sep=comma] {../CSV/figure_A3\cursf_panel4.csv};
\addplot +[only marks, color=blue, mark=star] table [x index=0, y index=1, col sep=comma] {../CSV/figure_A3\cursf_panel5.csv};
\addplot[no markers, very thick, color=black, dashed ] table [x index=0, y index=1, col sep=comma] {../CSV/figure_A3\cursf_panel1.csv};
\addplot[no markers, very thick, color=black] table [x index=0, y index=1, col sep=comma] {../CSV/figure_A3\cursf_panel2.csv};

\end{axis}

\end{tikzpicture}
}

\end{subfigure}
%%%%%%%%%%%%%%%%%%%%%%%%%%%%%%%%%%%%%%%%%%%%%%%%%%%%%%%%%%%%%%%%%%%%%%%%%%%%%%%%%%%%%%%%%%%%%%%%%%%%%%%%%%%%%%%%%%%%%%%%%%%%%%%%%%%%%%%%%%%
    %%%%%%%%%%%%%%%%%%%%%%%%%%%%%%%%%%%%%%%%%%%%% FIGURE A3 B %%%%%%%%%%%%%%%%%%%%%%%%%%%%%%%%%%%%%%%%%%%%%%%%%%%%%%%%%%%%%%%%%%%%%%%%%%%%%%%%%%
    %%%%%%%%%%%%%%%%%%%%%%%%%%%%%%%%%%%%%%%%%%%%%%%%%%%%%%%%%%%%%%%%%%%%%%%%%%%%%%%%%%%%%%%%%%%%%%%%%%%%%%%%%%%%%%%%%%%%%%%%%%%%%%%%%%%%%%%%%%%
\hspace{0.1cm}
    \begin{subfigure}[b]{0.5\linewidth}
    \centering
    \resizebox{\linewidth}{!}{

\begin{tikzpicture}

\begin{axis}
       [
         axis line style = { draw = none },
         %ymode=log,
         xtick           = {0, 1}         ,
         ytick           = {-10, 0, 10}         ,
         ymin = -14,
         ymax = 14 ,
         tick pos        = left           ,
         xlabel={},
         xticklabels     = {{Base}, {Test}},
         ylabel = {$\log_{10}(E)$},
       ]
       

\def\cursf{b}

\addplot +[only marks, color=black, mark=star] table [x index=0, y index=1, col sep=comma] {../CSV/figure_A3\cursf_panel3.csv};
\addplot +[only marks, color=red, mark=star] table [x index=0, y index=1, col sep=comma] {../CSV/figure_A3\cursf_panel4.csv};
\addplot +[only marks, color=blue, mark=star] table [x index=0, y index=1, col sep=comma] {../CSV/figure_A3\cursf_panel5.csv};
\addplot[no markers, very thick, color=black, dashed ] table [x index=0, y index=1, col sep=comma] {../CSV/figure_A3\cursf_panel1.csv};
\addplot[no markers, very thick, color=black] table [x index=0, y index=1, col sep=comma] {../CSV/figure_A3\cursf_panel2.csv};

\end{axis}

\end{tikzpicture}
}

\end{subfigure}
%%%%%%%%%%%%%%%%%%%%%%%%%%%%%%%%%%%%%%%%%%%%%%%%%%%%%%%%%%%%%%%%%%%%%%%%%%%%%%%%%%%%%%%%%%%%%%%%%%%%%%%%%%%%%%%%%%%%%%%%%%%%%%%%%%%%%%%%%%%
    %%%%%%%%%%%%%%%%%%%%%%%%%%%%%%%%%%%%%%%%%%%%% FIGURE A3 C %%%%%%%%%%%%%%%%%%%%%%%%%%%%%%%%%%%%%%%%%%%%%%%%%%%%%%%%%%%%%%%%%%%%%%%%%%%%%%%%%%
    %%%%%%%%%%%%%%%%%%%%%%%%%%%%%%%%%%%%%%%%%%%%%%%%%%%%%%%%%%%%%%%%%%%%%%%%%%%%%%%%%%%%%%%%%%%%%%%%%%%%%%%%%%%%%%%%%%%%%%%%%%%%%%%%%%%%%%%%%%%
%\hspace{0.1cm}
%    \begin{subfigure}[b]{0.5\linewidth}
%    \centering
%    \resizebox{\linewidth}{!}{
%
%\begin{tikzpicture}
%
%\begin{axis}
%       [
%         axis line style = { draw = none },
%         %ymode=log,
%         xtick           = {0, 1}         ,
%         ytick           = {-10, 0, 10}         ,
%         ymin = -14,
%         ymax = 14 ,
%         tick pos        = left           ,
%         xlabel={},
%         xticklabels     = {{Base}, {Test}},
%         ylabel = {$\log_{10}(C*E)$},
%       ]
%       
%\def\cursf{c}
%
%\addplot +[only marks, color=black, mark=star] table [x index=0, y index=1, col sep=comma] {../CSV/figure_A3\cursf_panel3.csv};
%\addplot +[only marks, color=red, mark=star] table [x index=0, y index=1, col sep=comma] {../CSV/figure_A3\cursf_panel4.csv};
%\addplot +[only marks, color=blue, mark=star] table [x index=0, y index=1, col sep=comma] {../CSV/figure_A3\cursf_panel5.csv};
%\addplot[no markers, very thick, color=black, dashed ] table [x index=0, y index=1, col sep=comma] {../CSV/figure_A3\cursf_panel1.csv};
%\addplot[no markers, very thick, color=black] table [x index=0, y index=1, col sep=comma] {../CSV/figure_A3\cursf_panel2.csv};
%
%
%\end{axis}
%
%\end{tikzpicture}
%}
%
%\end{subfigure}
%
%\vspace{2cm}

%%%%%%%%%%%%%%%%%%%%%%%%%%%%%%%%%%%%%%%%%%%%%%%%%%%%%%%%%%%%%%%%%%%%%%%%%%%%%%%%%%%%%%%%%%%%%%%%%%%%%%%%%%%%%%%%%%%%%%%%%%%%%%%%%%%%%%%%%%%
    %%%%%%%%%%%%%%%%%%%%%%%%%%%%%%%%%%%%%%%%%%%%% FIGURE A3 D %%%%%%%%%%%%%%%%%%%%%%%%%%%%%%%%%%%%%%%%%%%%%%%%%%%%%%%%%%%%%%%%%%%%%%%%%%%%%%%%%%
    %%%%%%%%%%%%%%%%%%%%%%%%%%%%%%%%%%%%%%%%%%%%%%%%%%%%%%%%%%%%%%%%%%%%%%%%%%%%%%%%%%%%%%%%%%%%%%%%%%%%%%%%%%%%%%%%%%%%%%%%%%%%%%%%%%%%%%%%%%%

\vspace{2cm}

    \begin{subfigure}[b]{0.5\linewidth}
    \centering
    \resizebox{\linewidth}{!}{

\begin{tikzpicture}

\begin{axis}
       [
         axis line style = { draw = none },
         %ymode=log,
         xtick           = {0, 1}         ,
         ytick           = {-50, -40, -30, -20, -10, 0, 10}  ,
         ymin = -57,
         ymax = 15 ,
         tick pos        = left           ,
         xlabel={},
         xticklabels    = {{Base}, {Test}},
         ylabel = {$\log_{10}(L_0)$},
       ]
       
\def\cursf{d}

\addplot +[only marks, color=black, mark=star] table [x index=0, y index=1, col sep=comma] {../CSV/figure_A3\cursf_panel3.csv};
\addplot +[only marks, color=red, mark=star] table [x index=0, y index=1, col sep=comma] {../CSV/figure_A3\cursf_panel4.csv};
\addplot +[only marks, color=blue, mark=star] table [x index=0, y index=1, col sep=comma] {../CSV/figure_A3\cursf_panel5.csv};
\addplot[no markers, very thick, color=black, dashed ] table [x index=0, y index=1, col sep=comma] {../CSV/figure_A3\cursf_panel1.csv};
\addplot[no markers, very thick, color=black] table [x index=0, y index=1, col sep=comma] {../CSV/figure_A3\cursf_panel2.csv};

\end{axis}

\end{tikzpicture}
}

\end{subfigure}
%%%%%%%%%%%%%%%%%%%%%%%%%%%%%%%%%%%%%%%%%%%%%%%%%%%%%%%%%%%%%%%%%%%%%%%%%%%%%%%%%%%%%%%%%%%%%%%%%%%%%%%%%%%%%%%%%%%%%%%%%%%%%%%%%%%%%%%%%%%
    %%%%%%%%%%%%%%%%%%%%%%%%%%%%%%%%%%%%%%%%%%%%% FIGURE A3 E %%%%%%%%%%%%%%%%%%%%%%%%%%%%%%%%%%%%%%%%%%%%%%%%%%%%%%%%%%%%%%%%%%%%%%%%%%%%%%%%%%
    %%%%%%%%%%%%%%%%%%%%%%%%%%%%%%%%%%%%%%%%%%%%%%%%%%%%%%%%%%%%%%%%%%%%%%%%%%%%%%%%%%%%%%%%%%%%%%%%%%%%%%%%%%%%%%%%%%%%%%%%%%%%%%%%%%%%%%%%%%%
\hspace{0.1cm}
    \begin{subfigure}[b]{0.5\linewidth}
    \centering
    \resizebox{\linewidth}{!}{

\begin{tikzpicture}

\begin{axis}
       [
         axis line style = { draw = none },
         %ymode=log,
         xtick           = {0, 1}         ,
         ytick           = {-50, -40, -30, -20, -10, 0, 10}  ,
         ymin = -57,
         ymax = 15 ,
         tick pos        = left           ,
         xlabel={},
         xticklabels     = {{Base}, {Test}},
         ylabel = {$\log_{10}(D)$},
       ]
       
\def\cursf{e}

\addplot +[only marks, color=black, mark=star] table [x index=0, y index=1, col sep=comma] {../CSV/figure_A3\cursf_panel3.csv};
\addplot +[only marks, color=red, mark=star] table [x index=0, y index=1, col sep=comma] {../CSV/figure_A3\cursf_panel4.csv};
\addplot +[only marks, color=blue, mark=star] table [x index=0, y index=1, col sep=comma] {../CSV/figure_A3\cursf_panel5.csv};
\addplot[no markers, very thick, color=black, dashed ] table [x index=0, y index=1, col sep=comma] {../CSV/figure_A3\cursf_panel1.csv};
\addplot[no markers, very thick, color=black] table [x index=0, y index=1, col sep=comma] {../CSV/figure_A3\cursf_panel2.csv};


\end{axis}

\end{tikzpicture}
}

\end{subfigure}
%%%%%%%%%%%%%%%%%%%%%%%%%%%%%%%%%%%%%%%%%%%%%%%%%%%%%%%%%%%%%%%%%%%%%%%%%%%%%%%%%%%%%%%%%%%%%%%%%%%%%%%%%%%%%%%%%%%%%%%%%%%%%%%%%%%%%%%%%%%
    %%%%%%%%%%%%%%%%%%%%%%%%%%%%%%%%%%%%%%%%%%%%% FIGURE A3 F %%%%%%%%%%%%%%%%%%%%%%%%%%%%%%%%%%%%%%%%%%%%%%%%%%%%%%%%%%%%%%%%%%%%%%%%%%%%%%%%%%
    %%%%%%%%%%%%%%%%%%%%%%%%%%%%%%%%%%%%%%%%%%%%%%%%%%%%%%%%%%%%%%%%%%%%%%%%%%%%%%%%%%%%%%%%%%%%%%%%%%%%%%%%%%%%%%%%%%%%%%%%%%%%%%%%%%%%%%%%%%%
%\hspace{0.1cm}
%    \begin{subfigure}[b]{0.30\linewidth}
%    \centering
%    \resizebox{\linewidth}{!}{
%
%\begin{tikzpicture}
%
%\begin{axis}
%       [
%         axis line style = { draw = none },
%         %ymode=log,
%         xtick           = {0, 1}         ,
%         ytick           = {-50, -40, -30, -20, -10, 0, 10}  ,
%         ymin = -57,
%         ymax = 15 ,
%         tick pos        = left           ,
%         xlabel={},
%         xticklabels     = {{Base}, {Test}},
%         ylabel = {$\log_{10}(L_0^{1/4}*D)$},
%       ]
%       
%\def\cursf{f}
%
%\addplot +[only marks, color=black, mark=star] table [x index=0, y index=1, col sep=comma] {../CSV/figure_A3\cursf_panel3.csv};
%\addplot +[only marks, color=red, mark=star] table [x index=0, y index=1, col sep=comma] {../CSV/figure_A3\cursf_panel4.csv};
%\addplot +[only marks, color=blue, mark=star] table [x index=0, y index=1, col sep=comma] {../CSV/figure_A3\cursf_panel5.csv};
%\addplot[no markers, very thick, color=black, dashed ] table [x index=0, y index=1, col sep=comma] {../CSV/figure_A3\cursf_panel1.csv};
%\addplot[no markers, very thick, color=black] table [x index=0, y index=1, col sep=comma] {../CSV/figure_A3\cursf_panel2.csv};
%
%\end{axis}
%
%\end{tikzpicture}
%}
%
%\end{subfigure}
%
%\vspace{2cm}

 %%%%%%%%%%%%%%%%%%%%%%%%%%%%%%%%%%%%%%%%%%%%%%%%%%%%%%%%%%%%%%%%%%%%%%%%%%%%%%%%%%%%%%%%%%%%%%%%%%%%%%%%%%%%%%%%%%%%%%%%%%%%%%%%%%%%%%%%%%%
    %%%%%%%%%%%%%%%%%%%%%%%%%%%%%%%%%%%%%%%%%%%%% FIGURE A3 G %%%%%%%%%%%%%%%%%%%%%%%%%%%%%%%%%%%%%%%%%%%%%%%%%%%%%%%%%%%%%%%%%%%%%%%%%%%%%%%%%%
    %%%%%%%%%%%%%%%%%%%%%%%%%%%%%%%%%%%%%%%%%%%%%%%%%%%%%%%%%%%%%%%%%%%%%%%%%%%%%%%%%%%%%%%%%%%%%%%%%%%%%%%%%%%%%%%%%%%%%%%%%%%%%%%%%%%%%%%%%%%


%    \begin{subfigure}[b]{0.45\linewidth}
%    \centering
%    \resizebox{\linewidth}{!}{
%
%\begin{tikzpicture}
%
%\begin{axis}
%       [
%         axis line style = { draw = none },
%         %ymode=log,
%         xtick           = {0, 1}         ,
%         ytick           = {-10, 0, 10}         ,
%         ymin = -14,
%         ymax = 14 ,
%         tick pos        = left           ,
%         xlabel={},
%         xticklabels    = {{Base}, {Test}},
%         ylabel = {$\log_{10}(J_0)$},
%       ]
%       
%
%
%\def\cursf{g}
%
%\addplot +[only marks, color=black, mark=star] table [x index=0, y index=1, col sep=comma] {../CSV/figure_A3\cursf_panel3.csv};
%\addplot +[only marks, color=red, mark=star] table [x index=0, y index=1, col sep=comma] {../CSV/figure_A3\cursf_panel4.csv};
%\addplot +[only marks, color=blue, mark=star] table [x index=0, y index=1, col sep=comma] {../CSV/figure_A3\cursf_panel5.csv};
%\addplot[no markers, very thick, color=black, dashed ] table [x index=0, y index=1, col sep=comma] {../CSV/figure_A3\cursf_panel1.csv};
%\addplot[no markers, very thick, color=black] table [x index=0, y index=1, col sep=comma] {../CSV/figure_A3\cursf_panel2.csv};
%
%\end{axis}
%
%\end{tikzpicture}
%}
%
%\end{subfigure}
%%%%%%%%%%%%%%%%%%%%%%%%%%%%%%%%%%%%%%%%%%%%%%%%%%%%%%%%%%%%%%%%%%%%%%%%%%%%%%%%%%%%%%%%%%%%%%%%%%%%%%%%%%%%%%%%%%%%%%%%%%%%%%%%%%%%%%%%%%%%
%    %%%%%%%%%%%%%%%%%%%%%%%%%%%%%%%%%%%%%%%%%%%%% FIGURE A3 H %%%%%%%%%%%%%%%%%%%%%%%%%%%%%%%%%%%%%%%%%%%%%%%%%%%%%%%%%%%%%%%%%%%%%%%%%%%%%%%%%%
%    %%%%%%%%%%%%%%%%%%%%%%%%%%%%%%%%%%%%%%%%%%%%%%%%%%%%%%%%%%%%%%%%%%%%%%%%%%%%%%%%%%%%%%%%%%%%%%%%%%%%%%%%%%%%%%%%%%%%%%%%%%%%%%%%%%%%%%%%%%%
%\hspace{0.1cm}
%    \begin{subfigure}[b]{0.45\linewidth}
%    \centering
%    \resizebox{\linewidth}{!}{
%
%\begin{tikzpicture}
%
%\begin{axis}
%       [
%         axis line style = { draw = none },
%         %ymode=log,
%         xtick           = {0, 1}         ,
%         ytick           = {-10, 0, 10}         ,
%         ymin = -14,
%         ymax = 14 ,
%         tick pos        = left           ,
%         xlabel={},
%         xticklabels     = {{Base}, {Test}},
%         ylabel = {$\log_{10}(zJ)$},
%       ]
%       
%
%\def\cursf{h}
%
%\addplot +[only marks, color=black, mark=star] table [x index=0, y index=1, col sep=comma] {../CSV/figure_A3\cursf_panel3.csv};
%\addplot +[only marks, color=red, mark=star] table [x index=0, y index=1, col sep=comma] {../CSV/figure_A3\cursf_panel4.csv};
%\addplot +[only marks, color=blue, mark=star] table [x index=0, y index=1, col sep=comma] {../CSV/figure_A3\cursf_panel5.csv};
%\addplot[no markers, very thick, color=black, dashed ] table [x index=0, y index=1, col sep=comma] {../CSV/figure_A3\cursf_panel1.csv};
%\addplot[no markers, very thick, color=black] table [x index=0, y index=1, col sep=comma] {../CSV/figure_A3\cursf_panel2.csv};
%
%\end{axis}
%
%\end{tikzpicture}
%}
%
%\end{subfigure}
%%%%%%%%%%%%%%%%%%%%%%%%%%%%%%%%%%%%%%%%%%%%%%%%%%%%%%%%%%%%%%%%%%%%%%%%%%%%%%%%%%%%%%%%%%%%%%%%%%%%%%%%%%%%%%%%%%%%%%%%%%%%%%%%%%%%%%%%%%%%
%    %%%%%%%%%%%%%%%%%%%%%%%%%%%%%%%%%%%%%%%%%%%%% FIGURE A3 I %%%%%%%%%%%%%%%%%%%%%%%%%%%%%%%%%%%%%%%%%%%%%%%%%%%%%%%%%%%%%%%%%%%%%%%%%%%%%%%%%%
%    %%%%%%%%%%%%%%%%%%%%%%%%%%%%%%%%%%%%%%%%%%%%%%%%%%%%%%%%%%%%%%%%%%%%%%%%%%%%%%%%%%%%%%%%%%%%%%%%%%%%%%%%%%%%%%%%%%%%%%%%%%%%%%%%%%%%%%%%%%%
%\hspace{0.1cm}
%    \begin{subfigure}[b]{0.45\linewidth}
%    \centering
%    \resizebox{\linewidth}{!}{
%
%\begin{tikzpicture}
%
%\begin{axis}
%       [
%         axis line style = { draw = none },
%         %ymode=log,
%         xtick           = {0, 1}         ,
%         ytick           = {-10, 0, 10}         ,
%         ymin = -14,
%         ymax = 14 ,
%         tick pos        = left           ,
%         xlabel={},
%         xticklabels     = {{Base}, {Test}},
%         ylabel = {$\log_{10}(K_d)$},
%       ]
%       
%\def\cursf{i}
%
%\addplot +[only marks, color=black, mark=star] table [x index=0, y index=1, col sep=comma] {../CSV/figure_A3\cursf_panel3.csv};
%\addplot +[only marks, color=red, mark=star] table [x index=0, y index=1, col sep=comma] {../CSV/figure_A3\cursf_panel4.csv};
%\addplot +[only marks, color=blue, mark=star] table [x index=0, y index=1, col sep=comma] {../CSV/figure_A3\cursf_panel5.csv};
%\addplot[no markers, very thick, color=black, dashed ] table [x index=0, y index=1, col sep=comma] {../CSV/figure_A3\cursf_panel1.csv};
%\addplot[no markers, very thick, color=black] table [x index=0, y index=1, col sep=comma] {../CSV/figure_A3\cursf_panel2.csv};
%
%
%\end{axis}
%
%\end{tikzpicture}
%}
%
%\end{subfigure}

\captionsetup{labelformat=empty,width=1.1\linewidth}
\caption{Figure 8 - supplement figure 2: Inferred parameters from the reduced $\lpo$ model. Values inferred by fitting the once-reduced $\lpo$ model to 100 noisy synthetic measurements generated from base (black) and test (red, blue) parameter sets..  Test parameters are different from those in Figure 8.  Here, the inferred values of our compensatory parameters fall into two modes.  This is because the model is degenerate at the true `best fit' parameter values that the noisy synthetic data were generated from.  Each set of inferred parameters was labeled either red or blue.  The two clouds therefore show that the parameter compensation is consistent for all fits.  Each cloud is discernible from the base parameters, and the mean of the values is discernible from the base parameters.}



\end{figure}

\addtocounter{figure}{-1}

\begin{figure}
\hspace{-1.5cm}
\begin{subfigure}[b]{0.65\linewidth}
	\resizebox{\linewidth}{!}{
		\begin{tikzpicture}
            \node[draw=none, fill=none] at (-1.3,6.3) {\Large A};
			\begin{axis}[
						axis x line = bottom,
			axis y line = left,
			axis line style={-},
				xmin = 2,
				xmax = 4,
				ymin = 2,
				ymax = 80,
				ylabel = {$p_{50}$},
				xlabel = {$n$},
				xtick pos=left,
				ytick={10,20,40,80},
				xtick={2,2.5,3,3.5,4},
				ytick pos=left,
				%yticklabels={\textcolor{red}{-10.6},\textcolor{black}{-7.6},\textcolor{red}{3.3},\textcolor{black}{5.3}},
				%xmajorticks=false,
				%yticklabels{}
			]
			\addplot[color=black,mark=*] plot [error bars/.cd, y dir = both, x dir = both, x explicit, y explicit] table[x index=3, y index=1, col sep=comma, y error index = 2, x error index = 4, only marks] {../CSV/clean_ph.csv};
			%\addplot[color=red,mark=*] table[x index=0, y index=1, col sep=comma, only marks] {../CSV/figure_3c_panel1_orig.csv};
			%\draw[scale=1,domain=0:8,dashed,variable=\x,black] plot ({\x},{-4.7+0*\x});
			%\draw[scale=1,domain=0:8,smooth,variable=\x,red] plot ({\x},{.8488*\x-5.2838});
			
			
			\end{axis}
			
	
		\end{tikzpicture}
	}
	\end{subfigure}
\begin{subfigure}[b]{0.65\linewidth}
	\resizebox{\linewidth}{!}{
		\begin{tikzpicture}
            \node[draw=none, fill=none] at (-1.3,6.3) {\Large B};
			\begin{axis}[
						axis x line = bottom,
			axis y line = left,
			axis line style={-},
				xmin = 2,
				xmax = 4,
				ymin = 2,
				ymax = 80,
				ytick={10,20,40,80},
				xtick={2,2.5,3,3.5,4},
				ylabel = {$p_{50}$},
				xlabel = {$n$},
				xtick pos=left,
				%ytick={-10.6,-7.6,3.3,5.3},
				ytick pos=left,
				%yticklabels={\textcolor{red}{-10.6},\textcolor{black}{-7.6},\textcolor{red}{3.3},\textcolor{black}{5.3}},
				%xmajorticks=false,
				%yticklabels{}
			]
			\addplot[color=black,mark=*] plot [error bars/.cd, y dir = both, x dir = both, x explicit, y explicit] table[x index=3, y index=1, x error index = 4, y error index=2, col sep=comma, only marks] {../CSV/clean_evolution.csv};
			%\addplot[color=red,mark=*] table[x index=0, y index=1, col sep=comma, only marks] {../CSV/figure_3c_panel1_orig.csv};
			%\draw[scale=1,domain=0:8,dashed,variable=\x,black] plot ({\x},{-4.7+0*\x});
			%\draw[scale=1,domain=0:8,smooth,variable=\x,red] plot ({\x},{.8488*\x-5.2838});
			
			
			\end{axis}
			
	
		\end{tikzpicture}
	}
	\end{subfigure}
	
	\vspace{1cm}
	
	\hspace{-1.5cm}
	\begin{subfigure}[b]{0.65\linewidth}
	\resizebox{\linewidth}{!}{
		\begin{tikzpicture}
            \node[draw=none, fill=none] at (-1.3,6.3) {\Large C};
			\begin{axis}[
			axis x line = bottom,
			axis y line = left,
			axis line style={-},
				%xmax = -0.5,
				%xmin = -1.3,
				%axis y discontinuity=crunch,
				enlargelimits=false,
				ymin = 0.9,
				ymax = 2,
				xmin = -1.4,
				xmax = -0.2,
				x dir=reverse,
				ylabel = {$\log_{10}(\rm L_{T0R4})$},
				xlabel = {$\log_{10}(\rm L_4^{1/4})$},
				xtick pos=left,
				ytick={1,1.2,1.4,1.6,1.8},
				xtick={-0.4,-0.6,-0.8,-1,-1.2},
				ytick pos=left,
				%yticklabels={\textcolor{red}{-10.6},\textcolor{black}{-7.6},\textcolor{red}{3.3},\textcolor{black}{5.3}},
				%xmajorticks=false,
				%yticklabels{}
			]
			\addplot[color=black,mark=*] plot [error bars/.cd, y dir = both, x dir = both, x explicit, y explicit] table[x index=7, y index=5, x error index = 8, y error index = 6, col sep=comma, only marks] {../CSV/clean_ph.csv};
			%\addplot[color=red,mark=*] table[x index=0, y index=1, col sep=comma, only marks] {../CSV/figure_3c_panel1_orig.csv};
			%\draw[scale=1,domain=0:8,dashed,variable=\x,black] plot ({\x},{-4.7+0*\x});
			%\draw[scale=1,domain=0:8,smooth,variable=\x,red] plot ({\x},{.8488*\x-5.2838});
			
			
			\end{axis}
			
	
		\end{tikzpicture}
	}
	\end{subfigure}
\begin{subfigure}[b]{0.65\linewidth}
	\resizebox{\linewidth}{!}{
		\begin{tikzpicture}
            \node[draw=none, fill=none] at (-1.3,6.3) {\Large D};
			\begin{axis}[
			axis x line = bottom,
			axis y line = left,
			axis line style={-},
				%xmax = -0.5,
				%xmin = -1.3,
				%axis y discontinuity=crunch,
				enlargelimits=false,
				ymin = 0.9,
				ymax = 2,
				xmin = -1.4,
				xmax = -0.2,
				x dir=reverse,
				ylabel = {$\log_{10}(\rm L_{T0R4})$},
				xlabel = {$\log_{10}(\rm L_4^{1/4})$},
				xtick pos=left,
				ytick={1,1.2,1.4,1.6,1.8},
				xtick={-0.4,-0.6,-0.8,-1,-1.2},
				ytick pos=left,
				%yticklabels={\textcolor{red}{-10.6},\textcolor{black}{-7.6},\textcolor{red}{3.3},\textcolor{black}{5.3}},
				%xmajorticks=false,
				%yticklabels{}
			]
			\addplot[color=black,mark=*] plot [error bars/.cd, y dir = both, x dir = both, x explicit, y explicit] table[x index=7, y index=5, x error index = 8, y error index=6, col sep=comma, only marks] {../CSV/clean_evolution.csv};
			%\addplot[color=red,mark=*] table[x index=0, y index=1, col sep=comma, only marks] {../CSV/figure_3c_panel1_orig.csv};
			%\draw[scale=1,domain=0:8,dashed,variable=\x,black] plot ({\x},{-4.7+0*\x});
			%\draw[scale=1,domain=0:8,smooth,variable=\x,red] plot ({\x},{.8488*\x-5.2838});
			
			
			\end{axis}
			
	
		\end{tikzpicture}
	}
	\end{subfigure}
	
\captionsetup{width=1.3\linewidth}
\caption{Analysis of hemoglobin oxygen saturation curves.  These figures were generated with data published by Milo et al.  The Figures aim to reproduce Milo et al's figure 2a,b (our Figure 8A,B) and Milo et al's figure 4a,b (our Figure 8C,D).  Each point in Figures 8A,C corresponds to human hemoglobin at a different pH.  Each point in Figures 8B,D corresponds to hemoglobin from a different mammal, at the same physiological condition.  We refer the reader to the appropriate figures in Milo et al. for full descriptions of pH conditions and mammals used. }
\end{figure}



\end{document}










